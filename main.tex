\documentclass[12pt,a4paper]{article}

% --------------------
% Packages
% --------------------
\usepackage[utf8]{inputenc}     % Encoding
\usepackage[T1]{fontenc}        % Font encoding
\usepackage{mathptmx}           % Times New Roman (preferred over times)
\usepackage{setspace}           % Line spacing
\usepackage{geometry}           % Page margins
\usepackage{titlesec}           % Custom section fonts
\usepackage{enumitem}           % Better lists
\usepackage{graphicx}           % For images
\usepackage{float}              % For H placement specifier
\usepackage{hyperref}           % For hyperlinks
\usepackage{xcolor}             % For colors
\usepackage{fancyvrb}           % Better verbatim environments
\usepackage{booktabs}           % Better tables
\usepackage{caption}            % Better captions
\usepackage{algorithm}          % Algorithm environment
\usepackage{algpseudocode}      % Pseudocode formatting

% --------------------
% Page Setup
% --------------------
\geometry{
    left=1in,
    right=1in,
    top=1in,
    bottom=1in
}

\setlength{\parindent}{0.5in}
\onehalfspacing

% --------------------
% Hyperref Setup
% --------------------
\hypersetup{
    colorlinks=true,
    linkcolor=blue!70!black,
    urlcolor=blue!70!black,
    citecolor=blue!70!black
}

% --------------------
% Section Formatting
% --------------------
\titleformat{\section}
{\fontsize{16}{18}\selectfont\bfseries}
{\thesection}{1em}{}

\titleformat{\subsection}
{\fontsize{14}{16}\selectfont\bfseries}
{\thesubsection}{1em}{}

\titleformat{\subsubsection}
{\fontsize{12}{14}\selectfont\bfseries}
{\thesubsubsection}{1em}{}

\titlespacing*{\section}{0pt}{2ex plus 1ex minus .2ex}{1.5ex plus .2ex}
\titlespacing*{\subsection}{0pt}{1.5ex plus 1ex minus .2ex}{1ex plus .2ex}
\titlespacing*{\subsubsection}{0pt}{1ex plus 1ex minus .2ex}{0.5ex plus .2ex}

% --------------------
% Document Start
% --------------------
\begin{document}

% ====================
% Title Page (Optional)
% ====================
    \begin{titlepage}
        \centering
        \vspace*{2cm}
        {\Huge\bfseries DevSync\par}
        \vspace{0.5cm}
        {\Large\itshape Java Code Quality Analysis Platform\par}
        \vspace{2cm}
        {\large Software Design Document\par}
        \vspace{1cm}
        {\large Version 1.0\par}
        \vfill
        {\large \today\par}
    \end{titlepage}

% ====================
% Table of Contents
% ====================
    \tableofcontents
    \newpage

% ====================
    \section{Introduction}
% ====================

    DevSync is a comprehensive Java code quality analysis platform that combines advanced Abstract Syntax Tree (AST) parsing with AI-powered recommendations to detect code smells, security vulnerabilities, and maintainability issues in Java projects. Built with a modern React+Vite frontend and Spring Boot backend, the platform serves as an automated code review assistant, enabling development teams to maintain high code quality standards throughout the software development lifecycle.

    \subsection{Project Scope}

    The project encompasses multiple interconnected modules designed to provide end-to-end code quality assessment capabilities.

    \subsubsection{Completed Modules}

    \begin{description}[font=\bfseries, leftmargin=0.5in]
        \item[User Authentication System]
        Secure user registration, login, and session management with BCrypt password encryption, ensuring data protection and user privacy compliance. Integrated with React Router DOM for seamless navigation and protected routes.

        \item[File Upload and Processing]
        ZIP/JAR file upload with drag-and-drop support, extraction, and Java file collection capabilities supporting project structures up to 50MB. Real-time upload progress tracking with visual feedback.

        \item[Advanced Code Analysis Engine]
        Twelve sophisticated detectors implementing complex algorithms for comprehensive code smell detection:
        \begin{itemize}[leftmargin=0.3in, font=\normalfont]
            \item Long Method Detector (cyclomatic and cognitive complexity)
            \item Code Duplication Detector
            \item Broken Modularization Detector (cohesion analysis)
            \item Complex Conditional Detector
            \item Deficient Encapsulation Detector
            \item Empty Catch Block Detector
            \item Long Identifier Detector
            \item Long Parameter List Detector
            \item Long Statement Detector
            \item Magic Number Detector
            \item Missing Default Case Detector
            \item Unnecessary Abstraction Detector
        \end{itemize}

        \item[AI Integration Module]
        Ollama AI service integration with fallback mechanisms for intelligent code recommendations. Provides context-aware suggestions for refactoring and code improvement when external services are unavailable.

        \item[Report Generation System]
        Comprehensive text-based reporting with severity categorization (Critical, Warning, Suggestion), visual diagram generation using PlantUML, and PDF export capabilities using jsPDF. Reports include executive summaries, detailed issue breakdowns, and actionable recommendations.

        \item[Analysis History Tracking]
        Persistent storage and retrieval of analysis sessions with detailed metrics, enabling trend analysis, code quality improvement tracking, and comparative analysis across multiple project versions.

        \item[Web-based Dashboard]
        Modern React 18.3.1 + Vite 7.1.7 frontend featuring:
        \begin{itemize}[leftmargin=0.3in, font=\normalfont]
            \item Radix UI component library for accessible, production-ready components
            \item Tailwind CSS v4 for utility-first styling with PostCSS optimization
            \item next-themes for seamless dark/light mode switching
            \item Recharts for interactive data visualization and analytics
            \item Framer Motion for smooth animations and transitions
            \item React Hook Form for efficient form validation
            \item Lucide React icons for consistent iconography
            \item Responsive design optimized for desktop, tablet, and mobile devices
        \end{itemize}

        \item[Admin Panel]
        Administrative interface for user management, system monitoring, configuration control, and analytics dashboard with real-time metrics visualization.
    \end{description}

    \subsubsection{Key Features Implemented}

    The platform incorporates several advanced features that distinguish it from conventional static analysis tools:

    \begin{description}[font=\bfseries, leftmargin=0.5in]
        \item[Multi-dimensional Code Quality Assessment]
        Utilizes both cyclomatic and cognitive complexity metrics to provide nuanced code quality evaluation beyond simple line counts.

        \item[Real-time AST Parsing]
        Integration with JavaParser library enables immediate Abstract Syntax Tree generation and analysis upon file upload, with progress tracking and visual feedback in the UI.

        \item[Severity-based Issue Classification]
        Three-tier classification system (Critical, Warning, Suggestion) prioritizes issues based on their impact on code quality and maintainability.

        \item[Historical Analysis Tracking]
        Comparison capabilities allow developers to track code quality improvements over time and identify regression patterns.

        \item[AI-enhanced Recommendations]
        Contextual suggestions powered by Ollama AI provide actionable guidance for addressing identified issues, with intelligent fallback mechanisms ensuring continuous service availability.

        \item[Secure File Handling]
        Access control mechanisms with JWT-based authentication ensure that uploaded code remains protected and accessible only to authorized users. Files are stored with unique identifiers and proper access validation.
    \end{description}

% ===================================================
    \section{Design Methodology and Software Process Model}
% ===================================================

    \subsection{Design Methodology: Object-Oriented Programming}

    The DevSync project follows Object-Oriented Programming (OOP) principles, providing a robust foundation for building maintainable and extensible software. This methodology was selected based on the following justifications:

    \subsubsection{Encapsulation}

    Each detector class, such as \texttt{LongMethodDetector} and \texttt{CodeDuplicationDetector}, encapsulates specific analysis logic and maintains internal state through private methods and fields. This design principle ensures data integrity by preventing unauthorized access to internal implementation details while providing clear, well-defined interfaces for module interaction. For example, the complexity calculation algorithms remain hidden within detector classes, exposing only the final analysis results through public methods.

    \subsubsection{Inheritance}

    The project utilizes inheritance through Spring Boot's component hierarchy and JPA entity relationships. All detectors share common analysis patterns inherited from base classes while extending functionality for specific detection scenarios. This hierarchical structure reduces code duplication and ensures consistent behavior across the detector family.

    \subsubsection{Polymorphism}

    The detector system demonstrates polymorphism where different detector implementations are processed uniformly through common interfaces. This enables flexible execution of the analysis pipeline, allowing new detectors to be added without modifying existing orchestration logic. The runtime binding of detector implementations supports extensibility and testing through mock implementations.

    \subsubsection{Abstraction}

    Complex analysis algorithms are abstracted into high-level interfaces, hiding implementation details while providing clear contracts for functionality. Users of the analysis engine interact with simplified APIs without needing to understand the underlying AST traversal algorithms or metric calculation formulas.

    \subsection{Software Process Model: Iterative and Incremental Development}

    The project follows an Iterative and Incremental Development model, chosen for its flexibility and risk mitigation capabilities in complex software projects.

    \subsubsection{Iterative Approach}

    The system was developed through multiple iterations, with each iteration focusing on specific functionality:

    \begin{enumerate}[leftmargin=0.7in]
        \item \textbf{Iteration 1:} Core authentication and user management
        \item \textbf{Iteration 2:} File processing and extraction pipeline
        \item \textbf{Iteration 3:} Basic code analysis with initial detectors
        \item \textbf{Iteration 4:} Report generation and visualization
        \item \textbf{Iteration 5:} AI integration and advanced recommendations
        \item \textbf{Iteration 6:} Admin panel and system monitoring
    \end{enumerate}

    This approach allowed continuous refinement and improvement of existing features based on feedback from each iteration.

    \subsubsection{Incremental Delivery}

    Features were delivered incrementally, starting with core functionality and progressively adding advanced capabilities. This enabled early testing and validation, ensuring that foundational components were stable before building dependent features.

    \subsubsection{Risk Mitigation}

    The iterative approach allowed early identification and resolution of technical challenges, particularly in AST parsing complexity and AI service integration. Prototyping during early iterations exposed potential issues before significant development investment.

    \subsubsection{Flexibility}

    The incremental model provided flexibility to adapt requirements based on testing feedback and emerging needs. Fallback mechanisms for AI service unavailability were added in response to reliability concerns identified during integration testing.

% ====================
    \section{System Overview}
% ====================

    DevSync is an intelligent Java code analysis platform designed to improve software quality through automated detection of code smells, security vulnerabilities, and maintainability issues. The system combines traditional static analysis techniques with modern AI-powered insights to provide comprehensive code quality assessment.

    \subsection{Functionality Context}

    The platform analyzes uploaded Java project files and generates detailed reports containing actionable recommendations. Users can track code quality improvements over time through historical analysis comparison and receive AI-generated suggestions for code enhancement. The workflow supports individual developers performing personal code reviews as well as team leads monitoring project-wide code quality metrics.

    \subsection{Design Context}

    The system architecture follows a client-server model where a React.js frontend communicates with a Spring Boot backend through RESTful APIs. The backend integrates multiple analysis engines and external AI services, providing a unified interface for code quality assessment regardless of the underlying analysis mechanism.

    \subsection{Background Information}

    Modern software development faces increasing complexity in maintaining code quality across large codebases. Manual code reviews are time-consuming and may miss subtle issues that automated tools can detect consistently. DevSync addresses this challenge by automating code analysis using advanced algorithms and AI-driven recommendations, reducing the burden on human reviewers while improving detection coverage.

% ===========================
    \subsection{Architectural Design}
% ===========================

    \subsubsection{Modular Program Structure}

    The DevSync system is decomposed into four major modules, each responsible for distinct aspects of the platform's functionality.

    \paragraph{Frontend Module (React+Vite)}

    The presentation layer consists of React components organized by functionality, built with modern tooling:

    \begin{description}[font=\bfseries, leftmargin=0.5in]
        \item[Authentication Components] Handle user login, registration, and session management with secure token storage using React Router DOM for navigation.

        \item[Dashboard Components] Display analysis results with interactive visualizations using Recharts, historical trends, and quick-access functionality with smooth transitions powered by Framer Motion.

        \item[Reusable UI Components] Comprehensive component library built with Radix UI primitives including dialogs, dropdowns, accordions, tabs, tooltips, and more. Styled with Tailwind CSS for consistent design and full accessibility compliance.

        \item[Theme System] Dynamic dark/light mode switching using next-themes with persistent user preferences and seamless transitions.

        \item[API Integration Layer] Manages all backend communication using Axios (v1.12.2) with request interceptors, error handling, and response transformation.

        \item[Build System] Vite-powered development environment providing instant hot module replacement (HMR), optimized production builds, and fast development server.
    \end{description}

    \paragraph{Backend Module (Spring Boot)}

    The application layer implements business logic through a layered architecture:

    \begin{description}[font=\bfseries, leftmargin=0.5in]
        \item[Controller Layer] REST endpoints handling HTTP requests with validation and response formatting.

        \item[Service Layer] Business logic implementation with transaction management and cross-cutting concerns.

        \item[Repository Layer] Data access abstraction using Spring Data JPA with custom query methods.

        \item[Analysis Engine] Orchestrates detector execution and result aggregation.
    \end{description}

    \paragraph{Analysis Engine Module}

    The core processing component responsible for code quality assessment:

    \begin{description}[font=\bfseries, leftmargin=0.5in]
        \item[AST Parser] JavaParser integration for source code parsing and syntax tree generation.

        \item[Detector Components] Specialized analyzers for different code smell categories.

        \item[Report Generator] Formats analysis results into human-readable reports with visual diagrams.

        \item[AI Integration] Ollama service connection with graceful degradation for service unavailability.
    \end{description}

    \paragraph{Data Management Module}

    Persistence and storage management:

    \begin{description}[font=\bfseries, leftmargin=0.5in]
        \item[User Management] Account data, preferences, and access control.

        \item[Analysis History] Tracking of past analyses with comparison capabilities.

        \item[File Management] Secure storage and retrieval of uploaded projects and generated reports.
    \end{description}

    \subsubsection{Module Relationships and Collaboration}

    The system modules collaborate through well-defined interfaces to provide comprehensive code analysis functionality. The frontend communicates with the backend through RESTful APIs for authentication, file uploads, and result retrieval. Controllers delegate business logic to services, which interact with repositories for data persistence. The analysis engine coordinates detector execution, report generation, and AI service communication with fallback mechanisms.

    Data flows unidirectionally from user input through processing layers to storage, with query paths returning results through the same layered structure. This separation ensures that changes to one layer do not cascade unnecessarily to others.

    \textit{Note: The detailed box-and-line diagram is provided in Appendix A.}

    \subsubsection{Detailed Architecture Mapping}

    The system implements a four-tier architecture with clear responsibility boundaries:

    \paragraph{Client Tier (Presentation Layer)}

    \begin{description}[font=\bfseries, leftmargin=0.5in]
        \item[React Components] Functional components with hooks for state and lifecycle management, utilizing React 18.3.1 features.
        \item[State Management] Context API for global state with local component state for UI interactions, enhanced with React Hook Form for form validation.
        \item[API Client] Centralized Axios instance with interceptors for authentication and error handling.
        \item[UI Component Library] Radix UI primitives providing accessible, unstyled components including Accordion, Alert Dialog, Avatar, Checkbox, Dialog, Dropdown Menu, Navigation Menu, Popover, Progress, Radio Group, Scroll Area, Select, Slider, Switch, Tabs, Toggle, and Tooltip.
        \item[Styling System] Tailwind CSS v4.1.16 with PostCSS for utility-first styling, class-variance-authority for component variants, and tailwind-merge for class name optimization.
        \item[Theme System] next-themes library providing seamless dark/light mode switching with system preference detection and persistent storage.
        \item[Visualization] Recharts library for interactive data visualization and analytics dashboards.
        \item[Animation] Framer Motion for smooth, performant animations and transitions throughout the interface.
    \end{description}

    \paragraph{Server Tier (Application Layer)}

    \begin{description}[font=\bfseries, leftmargin=0.5in]
        \item[Web Layer] Spring MVC controllers with request validation and response serialization.
        \item[Business Layer] Service classes implementing domain logic with transactional boundaries.
        \item[Integration Layer] External service connectors for AI and potential future integrations.
        \item[Security Layer] Spring Security configuration with JWT-based authentication.
    \end{description}

    \paragraph{Data Tier (Persistence Layer)}

    \begin{description}[font=\bfseries, leftmargin=0.5in]
        \item[Repository Layer] JPA repositories with custom queries for complex data retrieval.
        \item[Entity Layer] Domain entities with JPA annotations and validation constraints.
        \item[Database Layer] MySQL database with optimized schema design.
        \item[File Storage] File system storage for uploaded projects and generated reports.
    \end{description}

    \paragraph{Analysis Tier (Processing Layer)}

    \begin{description}[font=\bfseries, leftmargin=0.5in]
        \item[Parser Layer] JavaParser integration for AST generation from source files.
        \item[Detection Layer] Individual detector implementations for specific code smells.
        \item[Analysis Layer] Orchestration of detector execution and result aggregation.
        \item[Reporting Layer] Report formatting and diagram generation.
    \end{description}

    This layered architecture ensures separation of concerns, maintainability, and scalability while providing clear interfaces between system components.

% ====================
    \section{Design Models}
% ====================

    \subsection{Activity Diagram: User Registration}
    This activity diagram explains how a user interacts with the system during registration and login. First, the user provides basic credentials. The system then checks whether the user already exists in the database. If the user is new, the registration form is displayed and the entered data goes through a validation process to ensure correctness. After successful validation, the user is granted access to the dashboard. In case of invalid data, the system stops the process and shows an error message.
    \begin{figure}[H]
        \centering
        \includegraphics[width=0.8\textwidth, keepaspectratio]{ActivityUserRegistration.jpeg}
        \caption{Activity Diagram for User Registration and Login}
        \label{fig:user-reg-activity}
    \end{figure}

    \subsection{Activity Diagram: Code Analysis}
    This diagram represents the main functionality of the DevSync system. The process starts when a developer uploads a ZIP file containing source code. The system extracts the files and runs multiple detectors to identify code issues such as smells or bad practices. These detectors work simultaneously to save time. After analysis, the system communicates with the local AI service (Ollama) to generate improvement suggestions. If the AI service is unavailable, the system follows a fallback mechanism to ensure that a report is still generated.
    \begin{figure}[H]
        \centering
        \includegraphics[width=0.8\textwidth, keepaspectratio]{ActivityCodeAnalysis.jpg}
        \caption{Activity Diagram for Code Analysis Workflow}
        \label{fig:code-analysis-activity}
    \end{figure}

    \subsection{Class Diagram: Whole System}
    The class diagram provides a structural overview of the DevSync system. It shows the main classes, their attributes, and methods, along with the relationships between them. Key components such as the UserController, UploadController, and ReportGenerator coordinate system behavior. The diagram also highlights how data is managed and stored in the database. Overall, this diagram acts as a blueprint that helps developers understand how different parts of the system are organized.
    \begin{figure}[H]
        \centering
        \includegraphics[width=0.8\textwidth, keepaspectratio]{classDiagram.jpg}
        \caption{Class Diagram for DevSync System}
        \label{fig:class-diagram}
    \end{figure}

    \subsection{Sequence Diagram: User Registration}
    This sequence diagram illustrates the step-by-step interaction involved in user registration and login. It shows how the user communicates with the graphical interface, which then forwards requests to the authentication controller. The controller validates the data by interacting with the database. An alternative (Alt) flow is included to handle both successful and failed login scenarios, clearly showing system behavior in each case.
    \begin{figure}[H]
        \centering
        \includegraphics[width=0.8\textwidth, keepaspectratio]{SD-userRegistration.jpeg}
        \caption{Sequence Diagram for User Registration}
        \label{fig:user-reg-seq}
    \end{figure}

    \subsection{Sequence Diagram: Code Analysis and AI Feedback}
    This sequence diagram focuses on the detailed workflow of code analysis. The UploadController manages the entire process by coordinating file extraction, detector execution, and communication with the AI service. Once analysis is complete, the generated report is saved in the system history and then displayed to the user. This diagram highlights how different components interact in a well-defined order to complete the analysis task.
    \begin{figure}[H]
        \centering
        \includegraphics[width=0.8\textwidth, keepaspectratio]{SD-UplaodController.jpeg}
        \caption{Detailed Sequence for Code Analysis and AI Integration}
        \label{fig:upload-controller-seq}
    \end{figure}

    \subsection{System Sequence Diagram (SSD)}
    The system sequence diagram presents a high-level view of the system from an external perspective. It treats the system as a black box and only shows interactions between the developer and the system. The diagram covers major events such as uploading source code, starting the analysis process, and retrieving past analysis results from history.
    \begin{figure}[H]
        \centering
        \includegraphics[width=0.8\textwidth, keepaspectratio]{systemsequancediagram.jpeg}
        \caption{System Sequence Diagram (High Level)}
        \label{fig:ssd}
    \end{figure}

    \subsection{State Transition Diagram: System States}
    This diagram describes how the system changes its state during execution. Initially, the system remains in an idle state. When a file is uploaded, it moves to the uploading state, followed by the analyzing state. The diagram also includes error states, such as invalid files or detector failures, and shows how the system either recovers from these errors or safely returns to the idle state.
    \begin{figure}[H]
        \centering
        \includegraphics[width=0.8\textwidth, keepaspectratio]{stateTransitionDiagram.jpg}
        \caption{State Transition Diagram of the System}
        \label{fig:state-transition}
    \end{figure}
% ====================
    \section{Data Design}
% ====================

    \subsection{Overview}

    DevSync is a code quality analysis system that transforms uploaded Java projects into structured analysis reports. The system processes source code files, detects code smells using twelve specialized detectors, and stores analysis results in a relational database while maintaining file-based report storage. This hybrid approach balances query performance with storage efficiency for large report files.

    \subsection{Data Architecture}

    \subsubsection{High-Level Data Flow}

    The data transformation pipeline processes user uploads through multiple stages, each refining the data representation:

    \begin{Verbatim}[frame=single, fontsize=\small]
User Upload (ZIP/JAR) via React Frontend
        |
        v
Vite Dev Server / Production Build
        |
        v
Axios HTTP Request to Spring Boot Backend
        |
        v
File Extraction & Validation
        |
        v
Java Source Code Parsing (JavaParser AST)
        |
        v
Code Analysis Engine (12 Detectors)
        |
        v
Issue Detection & Scoring
        |
        v
Ollama AI Recommendations (with fallback)
        |
        v
Report Generation (TXT/Visual/PDF)
        |
        v
Database Persistence + File Storage
        |
        v
JSON Response to Frontend
        |
        v
Recharts Visualization & UI Presentation
    \end{Verbatim}

    \subsubsection{System Layers}

    The data architecture spans six distinct layers, each with specific responsibilities:

    \begin{enumerate}[leftmargin=0.7in]
        \item \textbf{Presentation Layer:} React 18.3.1 + Vite 7.1.7 frontend with Radix UI components, Tailwind CSS styling, and Recharts visualization. Handles user interaction, data visualization, and responsive design across devices.

        \item \textbf{Communication Layer:} Axios HTTP client with interceptors for authentication, error handling, and request/response transformation. Manages REST API communication between frontend and backend.

        \item \textbf{Application Layer:} Spring Boot controllers and services manage request routing, business orchestration, and transaction boundaries.

        \item \textbf{Business Logic Layer:} Code analysis engine with twelve specialized detectors implements core analysis algorithms, AST traversal, and metric calculation.

        \item \textbf{Data Access Layer:} JPA repositories provide abstracted database operations with transaction management and query optimization.

        \item \textbf{Persistence Layer:} MySQL database and file system storage maintain durable data records with proper indexing and access control.
    \end{enumerate}

    \subsection{Data Transformation and Processing}

    \subsubsection{Input Data Transformation}

    The system processes input data through six sequential stages, each transforming the data into increasingly refined representations.

    \paragraph{Stage 1: File Upload Processing}

    \textbf{Input:} Multipart file (ZIP/JAR format, maximum 50MB) uploaded via React frontend with drag-and-drop support or file picker

    \textbf{Process:}
    \begin{itemize}[leftmargin=0.5in]
        \item Frontend validates file type and size before upload
        \item Axios sends multipart/form-data request with progress tracking
        \item Backend extracts compressed files to a temporary directory
        \item Generate a unique folder name using timestamp and UUID
        \item Validate project file structure for expected Java project layout
        \item Verify file integrity and scan for malformed archives
        \item Real-time progress updates sent to frontend via WebSocket or polling
    \end{itemize}

    \textbf{Output:} Extracted project directory structure with validated contents and upload confirmation

    \paragraph{Stage 2: Source Code Parsing}

    \textbf{Input:} Java source files (.java extension)

    \textbf{Process:}
    \begin{itemize}[leftmargin=0.5in]
        \item JavaParser generates Abstract Syntax Tree (AST) for each source file
        \item CompilationUnit objects created containing parsed syntax structure
        \item AST nodes extracted including classes, methods, fields, and statements
        \item Symbol resolution performed for type reference analysis
    \end{itemize}

    \textbf{Output:} Structured CompilationUnit objects representing source code

    \paragraph{Stage 3: Code Analysis}

    \textbf{Input:} CompilationUnit AST objects

    \textbf{Process:}
    \begin{itemize}[leftmargin=0.5in]
        \item Twelve detector instances analyze AST in parallel where possible
        \item Detector-specific algorithms traverse relevant AST nodes
        \item Metrics calculated including LOC, complexity, cohesion, and coupling
        \item Threshold comparison identifies violations and generates issues
    \end{itemize}

    \textbf{Output:} Collection of CodeIssue objects with location and severity

    \paragraph{Stage 4: Issue Aggregation}

    \textbf{Input:} Raw issue strings from detector components

    \textbf{Process:}
    \begin{itemize}[leftmargin=0.5in]
        \item Parse issue format: \texttt{[Severity] [Type] File:Line - Message}
        \item Extract structured fields into typed objects
        \item Calculate severity distribution counts
        \item Compute grading metrics based on issue weights
    \end{itemize}

    \textbf{Output:} Structured analysis results map with aggregated statistics

    \paragraph{Stage 5: Report Generation}

    \textbf{Input:} Analysis results map with issue data and AI recommendations

    \textbf{Process:}
    \begin{itemize}[leftmargin=0.5in]
        \item Format issues into human-readable text with contextual information
        \item Generate PlantUML class and sequence diagrams
        \item Calculate summary statistics and overall quality grades
        \item Create comprehensive report with executive summary
        \item Generate PDF version using jsPDF for download capability
        \item Prepare JSON response with structured data for frontend visualization
    \end{itemize}

    \textbf{Output:} Text report file, visual diagram images, PDF document, and JSON data for Recharts visualization

    \paragraph{Stage 6: Data Persistence and Presentation}

    \textbf{Input:} Analysis results, metadata, and generated reports

    \textbf{Process:}
    \begin{itemize}[leftmargin=0.5in]
        \item Map results to JPA entity objects
        \item Validate constraints and referential integrity
        \item Execute database transactions with rollback support
        \item Store report files to designated file system location
        \item Send JSON response to frontend via Axios
        \item Frontend processes data for Recharts visualization
        \item Update UI with analysis results, charts, and downloadable reports
        \item Apply theme-aware styling using next-themes
        \item Animate transitions using Framer Motion
    \end{itemize}

    \textbf{Output:} Persisted database records, stored report files, and interactive dashboard with visualizations

    \subsubsection{Data Processing Algorithms}

    The analysis engine employs several key algorithms for code quality assessment:

    \paragraph{Cohesion Index Calculation}

    \begin{Verbatim}[frame=single, fontsize=\small]
cohesionIndex = usedFields / totalFields
    \end{Verbatim}

    This metric measures how comprehensively methods utilize class fields. Values range from 0.0 (low cohesion, indicating potential god class) to 1.0 (high cohesion, indicating focused responsibility). The system flags classes with cohesion index below 0.4 as exhibiting broken modularization.

    \paragraph{Coupling Count Calculation}

    \begin{Verbatim}[frame=single, fontsize=\small]
couplingCount = count(unique external dependencies)
    \end{Verbatim}

    This metric counts distinct external class references within a class. High coupling (values greater than 6) indicates excessive dependencies that reduce maintainability and increase change impact.

    \paragraph{Cyclomatic Complexity}

    \begin{Verbatim}[frame=single, fontsize=\small]
complexity = 1 + count(decision_points)
    \end{Verbatim}

    Cyclomatic complexity measures the number of independent execution paths through a method. Decision points include if statements, loops, case labels, and boolean operators. Methods with complexity above 10 are flagged as requiring refactoring.

    \paragraph{Cognitive Complexity}

    \begin{Verbatim}[frame=single, fontsize=\small]
cognitiveComplexity = sum(nesting_penalties) + sum(structural_penalties)
    \end{Verbatim}

    Cognitive complexity measures human comprehension difficulty, accounting for nesting depth and control flow breaks. Unlike cyclomatic complexity, it penalizes deeply nested structures more heavily. Values above 15 indicate hard-to-understand code requiring simplification.

    \paragraph{Issue Density}

    \begin{Verbatim}[frame=single, fontsize=\small]
issueDensity = totalIssues / (totalLOC / 1000)
    \end{Verbatim}

    Issue density normalizes the issue count per thousand lines of code, enabling fair comparison across projects of different sizes.

    \paragraph{Grading Algorithm}

    \begin{Verbatim}[frame=single, fontsize=\small]
score = 100 - (critical * 10 + warning * 5 + suggestion * 2)
    \end{Verbatim}

    The grading system assigns letter grades based on weighted issue counts:

    \begin{itemize}[leftmargin=0.5in]
        \item \textbf{A+ (95--100):} Excellent code quality with minimal issues
        \item \textbf{A (90--94):} Very good quality with few minor issues
        \item \textbf{B (80--89):} Good quality with some improvement areas
        \item \textbf{C (70--79):} Acceptable quality requiring attention
        \item \textbf{D (60--69):} Below average quality with significant issues
        \item \textbf{F (below 60):} Poor quality requiring major refactoring
    \end{itemize}

    \subsection{Data Storage}

    \subsubsection{Database Storage (MySQL)}

    The system uses MySQL as the primary relational database with the following configuration:

    \textbf{Database Name:} \texttt{devsyncdb}

    The database stores persistent system data including users, settings, and analysis history. Core tables include:

    \begin{description}[font=\bfseries, leftmargin=0.5in]
        \item[users] User account information including credentials and profile data
        \item[user\_settings] Per-user configuration preferences and notification settings
        \item[analysis\_history] Records of all analysis sessions with summary metrics
        \item[commit\_analysis] Git commit-level analysis results for repository tracking
        \item[admin\_settings] System-wide configuration parameters
    \end{description}

    \subsubsection{File System Storage}

    Report files and uploaded projects are stored in a structured file system hierarchy:

    \textbf{Base Directory:} \texttt{uploads/}

    \begin{Verbatim}[frame=single, fontsize=\small]
uploads/
  +-- {timestamp}/
  |     +-- {project_name}/
  |     +-- report.txt
  |     +-- diagrams/
  |            +-- class_diagram.png
  |            +-- sequence_diagram.png
    \end{Verbatim}

    Files are organized using timestamp-based unique folders to prevent naming collisions. Report file paths are stored in the database and served through backend controllers with appropriate access control.

    \subsubsection{In-Memory Processing}

    During analysis, temporary data structures are utilized for processing efficiency:

    \begin{itemize}[leftmargin=0.5in]
        \item Analysis results map (cleared after report generation to free memory)
        \item Singleton detector instances (reused across analyses for efficiency)
        \item AST objects retained only during the analysis lifecycle
        \item Caching of frequently accessed configuration values
    \end{itemize}

    \subsection{Summary}

    The DevSync data design implements a robust and scalable architecture that transforms uploaded source code into structured AST representations, processes them through twelve configurable detectors, and stores results using a hybrid database and file-based approach. The modern React 18.3.1 + Vite 7.1.7 frontend provides an intuitive, accessible interface with:

    \begin{itemize}[leftmargin=0.5in]
        \item Radix UI component library for production-ready, accessible components
        \item Tailwind CSS v4 for efficient, utility-first styling
        \item next-themes for seamless dark/light mode switching
        \item Recharts for interactive data visualization
        \item Framer Motion for smooth animations and transitions
        \item jsPDF for client-side PDF generation
        \item React Hook Form for efficient form validation
    \end{itemize}

    The design supports extensibility through modular detector architecture, efficient processing through parallel analysis where applicable, and long-term analysis tracking through persistent storage. The separation of transactional data (database) from large artifacts (file system) optimizes both query performance and storage efficiency while maintaining system maintainability. The frontend-backend separation enables independent scaling, technology updates, and enhanced user experience through modern web technologies.

% ====================
    \section{Human Interface Design}
% ====================

    \subsection{System Functionality from User Perspective}

    \subsubsection{Landing Page Experience}

    Users are greeted with a comprehensive landing page that showcases DevSync's capabilities through:

    \paragraph{Hero Section}
    Prominent call-to-action with clear value proposition - "Eliminate Java Code Smells" with gradient text effects and animated background elements that create visual engagement without distraction.

    \paragraph{Feature Showcase}
    Interactive sections highlighting key capabilities:
    \begin{itemize}[leftmargin=0.5in]
        \item AST-based code analysis with visual code examples
        \item AI-powered recommendations with mock analysis results
        \item Multi-dimensional quality assessment with metric explanations
        \item Historical tracking capabilities with trend visualization
    \end{itemize}

    \paragraph{How-to-Use Guide}
    Step-by-step process visualization:
    \begin{enumerate}[leftmargin=0.7in]
        \item Sign up and login with visual indicators
        \item Upload project with drag-and-drop demonstration
        \item Receive analysis with sample report preview
        \item Collaborate with team sharing features
    \end{enumerate}

    \paragraph{Theme Switching}
    Users can toggle between dark and light modes with smooth transitions, ensuring comfortable viewing in different environments and personal preferences.

    \subsubsection{Authentication Flow}

    \paragraph{Registration Process}
    Streamlined signup with real-time validation:
    \begin{itemize}[leftmargin=0.5in]
        \item Username availability checking with immediate feedback
        \item Email format validation with clear error messaging
        \item Password strength indicators with security requirements
        \item Success confirmation with automatic redirection
    \end{itemize}

    \paragraph{Login Interface}
    Secure authentication with user-friendly features:
    \begin{itemize}[leftmargin=0.5in]
        \item Email/password combination with remember me option
        \item Clear error messaging for invalid credentials
        \item Social login placeholders for future OAuth integration
        \item Password recovery options (planned enhancement)
    \end{itemize}

    \subsubsection{Main Dashboard Experience}

    \paragraph{Upload Interface}
    Intuitive file upload with multiple interaction methods:
    \begin{itemize}[leftmargin=0.5in]
        \item Drag-and-drop zone with visual feedback during hover and drop states
        \item Traditional file browser integration with file type filtering
        \item Upload progress indication with real-time status updates
        \item File validation with clear acceptance criteria (ZIP files only)
    \end{itemize}

    \paragraph{Analysis Progress}
    Real-time feedback during processing:
    \begin{itemize}[leftmargin=0.5in]
        \item Progress indicators showing current analysis stage
        \item Estimated completion time based on project size
        \item Cancellation option for long-running analyses
        \item Error handling with detailed failure explanations
    \end{itemize}

    \paragraph{Results Visualization}
    Comprehensive analysis presentation:
    \begin{itemize}[leftmargin=0.5in]
        \item Severity-based issue categorization with color coding (Critical, Warning, Suggestion)
        \item Numerical summaries with visual progress bars
        \item Detailed analysis summary in formatted text blocks
        \item Interactive report viewing with modal dialogs
    \end{itemize}

    \subsubsection{Historical Analysis Management}

    \paragraph{History Panel}
    Chronological analysis tracking:
    \begin{itemize}[leftmargin=0.5in]
        \item Timeline view of previous analyses with project names and dates
        \item Quick metrics overview for each analysis session
        \item Filtering and search capabilities for large analysis histories
        \item Comparison tools for tracking improvement trends
    \end{itemize}

    \paragraph{Report Access}
    Seamless report retrieval and viewing:
    \begin{itemize}[leftmargin=0.5in]
        \item One-click access to detailed analysis reports
        \item Full-screen report viewing with syntax highlighting
        \item Download options for offline review and sharing
        \item Print-friendly formatting for documentation purposes
    \end{itemize}

    \subsubsection{Administrative Interface}

    \paragraph{Admin Dashboard}
    Comprehensive system management:
    \begin{itemize}[leftmargin=0.5in]
        \item User account overview with registration statistics
        \item System performance metrics with real-time monitoring
        \item Analysis volume tracking with usage patterns
        \item Administrative controls for user management
    \end{itemize}

    \paragraph{User Management}
    Detailed user administration:
    \begin{itemize}[leftmargin=0.5in]
        \item User account listing with search and filter capabilities
        \item Account status management with activation/deactivation controls
        \item Usage statistics per user with analysis frequency tracking
        \item Security monitoring with login attempt tracking
    \end{itemize}

    \subsection{Feedback Information Display}

    \subsubsection{Success Feedback}

    \paragraph{Analysis Completion}
    Clear success indicators with actionable next steps:
    \begin{itemize}[leftmargin=0.5in]
        \item Completion notifications with issue count summaries
        \item Success animations with visual confirmation
        \item Immediate access to results with prominent view buttons
        \item Sharing options for team collaboration
    \end{itemize}

    \paragraph{User Actions}
    Positive reinforcement for user interactions:
    \begin{itemize}[leftmargin=0.5in]
        \item Registration success with welcome messaging
        \item Login confirmation with personalized greetings
        \item File upload success with processing initiation
        \item Settings updates with immediate effect confirmation
    \end{itemize}

    \subsubsection{Error Handling and Guidance}

    \paragraph{Validation Errors}
    Helpful error messaging with correction guidance:
    \begin{itemize}[leftmargin=0.5in]
        \item Form validation with field-specific error indicators
        \item File upload errors with format and size requirement explanations
        \item Authentication failures with clear resolution steps
        \item Network errors with retry mechanisms and offline indicators
    \end{itemize}

    \paragraph{System Errors}
    Graceful error handling with user-friendly explanations:
    \begin{itemize}[leftmargin=0.5in]
        \item Analysis failures with detailed error descriptions and suggested solutions
        \item Service unavailability notifications with estimated recovery times
        \item Data access errors with alternative action suggestions
        \item Timeout handling with automatic retry options
    \end{itemize}

    \subsubsection{Informational Feedback}

    \paragraph{Process Status}
    Transparent system state communication:
    \begin{itemize}[leftmargin=0.5in]
        \item Loading states with descriptive text and progress indicators
        \item Queue position for analysis processing during high load
        \item System maintenance notifications with scheduled downtime information
        \item Feature availability status with alternative options
    \end{itemize}

    \paragraph{Educational Content}
    Contextual help and guidance:
    \begin{itemize}[leftmargin=0.5in]
        \item Tooltips explaining technical terms and metrics
        \item Help sections with detailed feature explanations
        \item Best practices recommendations integrated into workflows
        \item Tutorial overlays for first-time users
    \end{itemize}

    \subsection{Screen Objects and Actions}

    \subsubsection{Navigation Objects and Actions}

    \paragraph{Header Navigation Bar}
    \textbf{Objects:} Logo, Theme Toggle Button, Admin Panel Button, Logout Button

    \textbf{Actions:}
    \begin{itemize}[leftmargin=0.5in]
        \item Logo Click: Context-dependent navigation
        \item Theme Toggle: Switch between dark and light modes with smooth transition animations
        \item Admin Panel: Navigate to admin login (if not authenticated) or admin dashboard (if authenticated)
        \item Logout: Clear user session, display confirmation toast, redirect to landing page
    \end{itemize}

    \paragraph{Sidebar Navigation (Dashboard)}
    \textbf{Objects:} Dashboard Link, New Analysis Link, History Link, Settings Link

    \textbf{Actions:}
    \begin{itemize}[leftmargin=0.5in]
        \item Dashboard: Return to main upload interface, reset any active analysis
        \item New Analysis: Clear current results, focus on upload area, display new analysis prompt
        \item History: Open history panel modal, load user's analysis history with pagination
        \item Settings: Open settings modal with user preferences and configuration options
    \end{itemize}

    \subsubsection{Upload Interface Objects and Actions}

    \paragraph{File Upload Area}
    \textbf{Objects:} Drag-and-drop Zone, File Browser Button, Selected File Display, Upload Progress Bar

    \textbf{Actions:}
    \begin{itemize}[leftmargin=0.5in]
        \item Drag Over: Highlight drop zone with visual feedback, show acceptance indicator
        \item File Drop: Validate file type and size, display file name, enable analyze button
        \item Click to Browse: Open file selection dialog, filter for ZIP files only
        \item File Selection: Update interface with selected file name, show file size and validation status
    \end{itemize}

    \paragraph{Analysis Control Buttons}
    \textbf{Objects:} Analyze Project Button, Cancel Analysis Button, New Analysis Button

    \textbf{Actions:}
    \begin{itemize}[leftmargin=0.5in]
        \item Analyze Project: Initiate file upload, start analysis pipeline, show progress indicators
        \item Cancel Analysis: Abort current analysis, clean up temporary files, return to upload state
        \item New Analysis: Reset interface, clear previous results, prepare for new file upload
    \end{itemize}

    \subsubsection{Results Display Objects and Actions}

    \paragraph{Metrics Cards}
    \textbf{Objects:} Critical Issues Card, Warnings Card, Suggestions Card

    \textbf{Actions:}
    \begin{itemize}[leftmargin=0.5in]
        \item Card Hover: Highlight card with subtle animation, show additional context
        \item Card Click: Filter detailed results by severity level, expand relevant sections
        \item Metric Animation: Count-up animation when results are first displayed
    \end{itemize}

    \paragraph{Analysis Summary Panel}
    \textbf{Objects:} Summary Text Area, Show Report Button, Download Button, Share Button

    \textbf{Actions:}
    \begin{itemize}[leftmargin=0.5in]
        \item Show Report: Open full report in modal dialog, format content for readability
        \item Download Report: Generate downloadable report file, trigger browser download
        \item Share Results: Generate shareable link, copy to clipboard with confirmation
    \end{itemize}

    \subsubsection{Report Viewing Objects and Actions}

    \paragraph{Report Modal Dialog}
    \textbf{Objects:} Report Content Area, Close Button, Search Box, Print Button

    \textbf{Actions:}
    \begin{itemize}[leftmargin=0.5in]
        \item Content Scroll: Smooth scrolling with syntax highlighting preservation
        \item Search: Highlight matching text, navigate between search results
        \item Print: Format report for printing, open print dialog with optimized layout
        \item Close: Save scroll position, close modal with fade animation
    \end{itemize}

    \paragraph{Report Navigation}
    \textbf{Objects:} Section Headers, Line Numbers, Issue Links, AI Analysis Section

    \textbf{Actions:}
    \begin{itemize}[leftmargin=0.5in]
        \item Section Click: Jump to specific report section, update navigation state
        \item Issue Link: Navigate to source code location, highlight problematic code
        \item AI Section Toggle: Expand/collapse AI analysis, show/hide recommendations
    \end{itemize}

    \subsubsection{History Management Objects and Actions}

    \paragraph{History List Interface}
    \textbf{Objects:} History Items, View Report Links, Delete Buttons, Filter Controls

    \textbf{Actions:}
    \begin{itemize}[leftmargin=0.5in]
        \item History Item Click: Expand item details, show analysis summary
        \item View Report: Load historical report, open in report modal
        \item Delete Analysis: Confirm deletion, remove from history, update display
        \item Filter/Sort: Apply filters, re-order list, update pagination
    \end{itemize}

    \paragraph{History Search and Filter}
    \textbf{Objects:} Search Input, Date Range Picker, Project Filter, Sort Dropdown

    \textbf{Actions:}
    \begin{itemize}[leftmargin=0.5in]
        \item Search Input: Real-time filtering, highlight matching results
        \item Date Range: Filter by analysis date, update results dynamically
        \item Project Filter: Group by project name, show project-specific history
        \item Sort Options: Re-order by date, issues count, or project name
    \end{itemize}

    \subsubsection{Administrative Interface Objects and Actions}

    \paragraph{User Management Panel}
    \textbf{Objects:} User List Table, Search Box, Action Buttons, Status Indicators

    \textbf{Actions:}
    \begin{itemize}[leftmargin=0.5in]
        \item User Row Click: Expand user details, show analysis statistics
        \item Search Users: Filter user list, highlight matching entries
        \item Status Toggle: Activate/deactivate user accounts, update status indicators
        \item View User Activity: Show detailed user analysis history and system usage
    \end{itemize}

    \paragraph{System Monitoring Dashboard}
    \textbf{Objects:} Metrics Cards, Performance Graphs, Alert Indicators, System Controls

    \textbf{Actions:}
    \begin{itemize}[leftmargin=0.5in]
        \item Metrics Refresh: Update real-time statistics, animate value changes
        \item Graph Interaction: Zoom into time periods, show detailed metrics
        \item Alert Click: Show alert details, provide resolution actions
        \item System Controls: Restart services, clear caches, update configurations
    \end{itemize}

    \subsubsection{Form Objects and Actions}

    \paragraph{Authentication Forms}
    \textbf{Objects:} Input Fields, Submit Buttons, Validation Messages, Social Login Buttons

    \textbf{Actions:}
    \begin{itemize}[leftmargin=0.5in]
        \item Input Focus: Show field requirements, clear previous errors
        \item Real-time Validation: Check input format, show validation status
        \item Form Submit: Validate all fields, show loading state, handle responses
        \item Social Login: Redirect to OAuth provider, handle authentication flow
    \end{itemize}

    \paragraph{Settings Forms}
    \textbf{Objects:} Preference Toggles, Configuration Inputs, Save Button, Reset Button

    \textbf{Actions:}
    \begin{itemize}[leftmargin=0.5in]
        \item Toggle Switch: Update preference immediately, show confirmation
        \item Input Change: Mark form as modified, enable save button
        \item Save Settings: Validate inputs, update user preferences, show success message
        \item Reset Form: Restore default values, clear modifications, confirm action
    \end{itemize}

    This comprehensive interface design ensures intuitive user interactions while providing powerful functionality for code analysis and system management. The design emphasizes clarity, feedback, and efficiency to support productive development workflows.


    \subsection{User Interface Screens}

    This section presents the user interface designs for both administrative and end-user application screens. All interfaces follow consistent design principles including responsive layouts, accessibility compliance, and theme support.

    \subsubsection{Admin Module Screens}

    The administrative interface provides system management capabilities for platform administrators.

    \begin{figure}[H]
        \centering
        \includegraphics[width=0.9\textwidth]{adashboard.png}
        \caption{Admin Dashboard---Main control panel providing an overview of system metrics, recent activities, and quick access to administrative functions.}
        \label{fig:admin-dashboard}
    \end{figure}

    \begin{figure}[H]
        \centering
        \includegraphics[width=0.9\textwidth]{adProjects.png}
        \caption{Admin Projects---Project management interface displaying all uploaded projects with filtering, sorting, and bulk action capabilities.}
        \label{fig:admin-projects}
    \end{figure}

    \begin{figure}[H]
        \centering
        \includegraphics[width=0.9\textwidth]{adReports.png}
        \caption{Admin Reports---Analytics and reporting screen showing aggregated code quality metrics across all projects and users.}
        \label{fig:admin-reports}
    \end{figure}

    \begin{figure}[H]
        \centering
        \includegraphics[width=0.9\textwidth]{adSetting.png}
        \caption{Admin Settings---System configuration options including analysis thresholds, AI service settings, and security policies.}
        \label{fig:admin-settings}
    \end{figure}

    \begin{figure}[H]
        \centering
        \includegraphics[width=0.9\textwidth]{aduser.png}
        \caption{Admin User Profile---Individual user profile view with account details, activity history, and administrative actions.}
        \label{fig:admin-user}
    \end{figure}

    \begin{figure}[H]
        \centering
        \includegraphics[width=0.9\textwidth]{adUsers.png}
        \caption{Admin Users---User list and management panel with search, filtering, and role assignment capabilities.}
        \label{fig:admin-users}
    \end{figure}

    \subsubsection{Application Screens}

    The end-user interface provides code analysis and reporting functionality for developers.

    \begin{figure}[H]
        \centering
        \includegraphics[width=0.9\textwidth]{Github.png}
        \caption{GitHub History---Version control activity log showing commit analysis results and code quality trends over repository history.}
        \label{fig:github-history}
    \end{figure}

    \begin{figure}[H]
        \centering
        \includegraphics[width=0.9\textwidth]{LandingPage.png}
        \caption{Landing Page---Initial application interface presenting platform features, benefits, and call-to-action elements.}
        \label{fig:landing-page}
    \end{figure}

    \begin{figure}[H]
        \centering
        \includegraphics[width=0.9\textwidth]{RegistrationPage.png}
        \caption{Registration Page---User account creation form with validation feedback and terms acceptance.}
        \label{fig:registration-page}
    \end{figure}

    \begin{figure}[H]
        \centering
        \includegraphics[width=0.9\textwidth]{setting.png}
        \caption{User Settings---Personal configuration screen for notification preferences, theme selection, and account management.}
        \label{fig:user-settings}
    \end{figure}

    \begin{figure}[H]
        \centering
        \includegraphics[width=0.9\textwidth]{uploadarea.png}
        \caption{Upload Area---File upload interface with drag-and-drop support, progress indication, and file validation feedback.}
        \label{fig:upload-area}
    \end{figure}

    \begin{figure}[H]
        \centering
        \includegraphics[width=0.9\textwidth]{useraccount.png}
        \caption{User Account---Profile and account details page showing usage statistics and subscription information.}
        \label{fig:user-account}
    \end{figure}

    \begin{figure}[H]
        \centering
        \includegraphics[width=0.9\textwidth]{WelcomeHomePage.png}
        \caption{Welcome Home Page---User dashboard after login displaying recent analyses, quick actions, and personalized recommendations.}
        \label{fig:welcome-home}
    \end{figure}
    \section{Implementation}

    This chapter discusses the implementation details of the DevSync code quality analysis system. The core modules are presented in pseudocode form following proper coding standards. The system implements a multi-detector architecture with AI-assisted analysis and GitHub integration.

    \subsection{Web Application Architecture}

    The DevSync platform is implemented as a full-stack web application with a clear separation between frontend and backend components. The frontend is built using React 18.3.1 with Vite 7.1.7 as the build tool, providing a modern, responsive user interface with hot module replacement during development. The backend utilizes Spring Boot framework with Java, offering RESTful API endpoints for all system operations.

    \subsubsection{Frontend Implementation}

    The React-based frontend leverages a component-driven architecture with functional components and React Hooks for state management. Key implementation features include:

    \begin{itemize}[leftmargin=0.5in]
        \item \textbf{Component Library:} Radix UI primitives provide accessible, unstyled components that are customized using Tailwind CSS utility classes for consistent styling across the application.
        \item \textbf{Styling System:} Tailwind CSS v4.1.16 with PostCSS enables utility-first styling with minimal custom CSS, ensuring responsive design and theme consistency.
        \item \textbf{Theme Management:} next-themes library implements seamless dark/light mode switching with persistent user preferences stored in local storage.
        \item \textbf{Data Visualization:} Recharts library renders interactive charts and graphs for analysis metrics, providing visual insights into code quality trends.
        \item \textbf{Animation Framework:} Framer Motion adds smooth transitions and animations throughout the interface, enhancing user experience without compromising performance.
        \item \textbf{Form Handling:} React Hook Form manages form state and validation efficiently, reducing re-renders and improving form performance.
        \item \textbf{HTTP Communication:} Axios v1.12.2 handles all API requests with interceptors for authentication token injection and centralized error handling.
        \item \textbf{Routing:} React Router DOM manages client-side routing with protected routes for authenticated users and role-based access control.
    \end{itemize}

    \subsubsection{Backend Implementation}

    The Spring Boot backend implements a layered architecture following industry best practices:

    \begin{itemize}[leftmargin=0.5in]
        \item \textbf{Controller Layer:} REST controllers handle HTTP requests, perform input validation using Bean Validation annotations, and return standardized JSON responses.
        \item \textbf{Service Layer:} Business logic is encapsulated in service classes with transactional boundaries managed by Spring's @Transactional annotation.
        \item \textbf{Repository Layer:} Spring Data JPA repositories provide database abstraction with custom query methods for complex data retrieval operations.
        \item \textbf{Security Layer:} Spring Security with JWT-based authentication secures API endpoints, with role-based authorization for administrative functions.
        \item \textbf{Analysis Engine:} Custom analysis engine orchestrates twelve specialized detector components, each implementing specific code smell detection algorithms.
        \item \textbf{AI Integration:} Service layer components integrate with Ollama AI service using HTTP client with fallback mechanisms for service unavailability.
        \item \textbf{File Management:} Multipart file upload handling with validation, extraction to temporary directories, and secure file storage with access control.
    \end{itemize}

    \subsubsection{Communication Protocol}

    The frontend and backend communicate through RESTful APIs using JSON data format. All API endpoints follow REST conventions with appropriate HTTP methods (GET, POST, PUT, DELETE) and status codes. Authentication is handled via JWT tokens passed in Authorization headers, with token refresh mechanisms for extended sessions. File uploads use multipart/form-data encoding with progress tracking support.

    \subsubsection{Deployment Configuration}

    The application supports both development and production environments. In development, Vite dev server runs on port 5173 with proxy configuration for backend API calls to Spring Boot running on port 8080. Production builds are optimized with code splitting, tree shaking, and asset minification. The backend can be deployed as a standalone JAR file or containerized using Docker for scalable deployment.

    \subsection{Code Analysis Engine}

    The Code Analysis Engine serves as the central orchestrator for all code quality detectors. It manages detector lifecycle, configuration, and result aggregation.

    \begin{algorithm}[H]
        \caption{analyzeProject}
        \textbf{Input:} projectPath (string), userSettings (UserSettings object)\\
        \textbf{Output:} analysisResults (Map containing issues, metrics, and statistics)
        \begin{algorithmic}[1]
            \State results $\gets$ empty Map
            \State allIssues $\gets$ empty List
            \State severityCounts $\gets$ empty Map
            \State detectorCounts $\gets$ empty Map
            \State configureDetectors(userSettings)
            \State javaFiles $\gets$ collectJavaFiles(projectPath)
            \State totalLOC $\gets$ 0, totalClasses $\gets$ 0, totalMethods $\gets$ 0
            \For{each file in javaFiles}
                \If{shouldExclude(file.path)}
                    \State continue
                \EndIf
                \State cu $\gets$ parseJavaFile(file)
                \If{cu is valid}
                    \State fileIssues $\gets$ analyzeFile(cu, file.name, detectorCounts)
                    \State allIssues.addAll(fileIssues)
                    \State updateSeverityCounts(fileIssues, severityCounts)
                    \State totalLOC $\gets$ totalLOC + countLinesOfCode(cu)
                    \State totalClasses $\gets$ totalClasses + countClasses(cu)
                    \State totalMethods $\gets$ totalMethods + countMethods(cu)
                \EndIf
            \EndFor
            \State results.put("issues", allIssues)
            \State results.put("severityCounts", severityCounts)
            \State results.put("detectorCounts", detectorCounts)
            \State results.put("totalLOC", totalLOC)
            \State results.put("totalClasses", totalClasses)
            \State results.put("totalMethods", totalMethods)
            \Return results
        \end{algorithmic}
    \end{algorithm}

    \begin{algorithm}[H]
        \caption{analyzeFile}
        \textbf{Input:} compilationUnit (AST), fileName (string), detectorCounts (Map)\\
        \textbf{Output:} issues (List of detected code smells)
        \begin{algorithmic}[1]
            \State issues $\gets$ empty List
            \For{each detector in enabledDetectors}
                \If{detector.isEnabled()}
                    \State detectorIssues $\gets$ detector.detect(compilationUnit)
                    \If{detectorIssues is not empty}
                        \State issues.addAll(detectorIssues)
                        \State detectorCounts.increment(detector.name, detectorIssues.size())
                    \EndIf
                \EndIf
            \EndFor
            \Return issues
        \end{algorithmic}
    \end{algorithm}

    \subsection{Long Method Detection}

    The Long Method Detector identifies methods that violate size and complexity thresholds using multiple metrics including cyclomatic complexity, cognitive complexity, and nesting depth.

    \begin{algorithm}[H]
        \caption{detectLongMethods}
        \textbf{Input:} compilationUnit (AST), thresholds (Configuration)\\
        \textbf{Output:} issues (List of long method violations)
        \begin{algorithmic}[1]
            \State issues $\gets$ empty List
            \State methods $\gets$ extractAllMethods(compilationUnit)
            \For{each method in methods}
                \State info $\gets$ analyzeMethod(method)
                \If{info.lineCount $>$ thresholds.baseLineThreshold OR
                \State \hspace{1cm} info.cyclomaticComplexity $>$ thresholds.maxCyclomaticComplexity OR
                \State \hspace{1cm} info.cognitiveComplexity $>$ thresholds.maxCognitiveComplexity OR
                \State \hspace{1cm} info.nestingDepth $>$ thresholds.maxNestingDepth}
                    \State score $\gets$ calculateComplexityScore(info)
                    \State severity $\gets$ mapScoreToSeverity(score)
                    \State issue $\gets$ formatIssue(severity, info, generateSuggestions(info))
                    \State issues.add(issue)
                \EndIf
            \EndFor
            \Return issues
        \end{algorithmic}
    \end{algorithm}

    \begin{algorithm}[H]
        \caption{calculateComplexityScore}
        \textbf{Input:} methodInfo (MethodInfo object)\\
        \textbf{Output:} score (double between 0.0 and 1.0)
        \begin{algorithmic}[1]
            \State lineScore $\gets$ min(1.0, methodInfo.lineCount / criticalThreshold)
            \State complexityScore $\gets$ min(1.0, methodInfo.cyclomaticComplexity / maxComplexity)
            \State cognitiveScore $\gets$ min(1.0, methodInfo.cognitiveComplexity / maxCognitive)
            \State nestingScore $\gets$ min(1.0, methodInfo.nestingDepth / maxNesting)
            \State responsibilityScore $\gets$ min(1.0, methodInfo.responsibilityCount / 3)
            \State methodType $\gets$ determineMethodType(methodInfo)
            \State weight $\gets$ getMethodTypeWeight(methodType)
            \State score $\gets$ weight $\times$ (lineScore $\times$ 0.35 + complexityScore $\times$ 0.25 +
            \State \hspace{2.5cm} cognitiveScore $\times$ 0.20 + nestingScore $\times$ 0.10 +
            \State \hspace{2.5cm} responsibilityScore $\times$ 0.10)
            \Return score
        \end{algorithmic}
    \end{algorithm}

    \begin{algorithm}[H]
        \caption{calculateCyclomaticComplexity}
        \textbf{Input:} methodDeclaration (AST node)\\
        \textbf{Output:} complexity (integer)
        \begin{algorithmic}[1]
            \State complexity $\gets$ 1
            \State complexity $\gets$ complexity + countNodes(methodDeclaration, IfStmt)
            \State complexity $\gets$ complexity + countNodes(methodDeclaration, ForStmt)
            \State complexity $\gets$ complexity + countNodes(methodDeclaration, WhileStmt)
            \State complexity $\gets$ complexity + countNodes(methodDeclaration, DoStmt)
            \State complexity $\gets$ complexity + countNodes(methodDeclaration, SwitchEntry)
            \State complexity $\gets$ complexity + countNodes(methodDeclaration, ConditionalExpr)
            \State complexity $\gets$ complexity + countLogicalOperators(methodDeclaration)
            \Return complexity
        \end{algorithmic}
    \end{algorithm}

    \subsection{Code Quality Grading System}

    The Grading System evaluates overall code quality using industry-standard metrics and issue density calculations.

    \begin{algorithm}[H]
        \caption{calculateGrade}
        \textbf{Input:} severityCounts (Map), totalLOC (integer)\\
        \textbf{Output:} gradeResult (GradeResult object)
        \begin{algorithmic}[1]
            \If{totalLOC $\leq$ 0}
                \Return GradeResult("N/A", 0, 0, "Cannot grade")
            \EndIf
            \State critical $\gets$ severityCounts.get("Critical")
            \State high $\gets$ severityCounts.get("High")
            \State medium $\gets$ severityCounts.get("Medium")
            \State low $\gets$ severityCounts.get("Low")
            \State totalIssues $\gets$ critical + high + medium + low
            \If{totalIssues = 0}
                \Return GradeResult("A+", 100.0, 0.0, "Perfect code quality")
            \EndIf
            \State weightedIssues $\gets$ (critical $\times$ 10.0) + (high $\times$ 5.0) + (medium $\times$ 2.0) + (low $\times$ 0.5)
            \State kloc $\gets$ totalLOC / 1000.0
            \State issueDensity $\gets$ totalIssues / kloc
            \State baseScore $\gets$ calculateBaseScore(issueDensity)
            \State finalScore $\gets$ applyPenalties(baseScore, critical, high, issueDensity)
            \State letterGrade $\gets$ mapScoreToGrade(finalScore)
            \State qualityLevel $\gets$ getQualityLevel(letterGrade)
            \State recommendation $\gets$ generateRecommendation(letterGrade, critical, high)
            \Return GradeResult(letterGrade, finalScore, issueDensity, qualityLevel, recommendation)
        \end{algorithmic}
    \end{algorithm}

    \begin{algorithm}[H]
        \caption{calculateBaseScore}
        \textbf{Input:} issueDensity (double, issues per KLOC)\\
        \textbf{Output:} score (double between 0 and 100)
        \begin{algorithmic}[1]
            \If{issueDensity $<$ 0.5}
                \Return 100 - (issueDensity / 0.5) $\times$ 10 \Comment{A range: 90-100}
            \ElsIf{issueDensity $<$ 2.0}
            \State ratio $\gets$ (issueDensity - 0.5) / (2.0 - 0.5)
            \Return 90 - (ratio $\times$ 10) \Comment{B range: 80-89}
            \ElsIf{issueDensity $<$ 5.0}
            \State ratio $\gets$ (issueDensity - 2.0) / (5.0 - 2.0)
            \Return 80 - (ratio $\times$ 10) \Comment{C range: 70-79}
            \ElsIf{issueDensity $<$ 10.0}
            \State ratio $\gets$ (issueDensity - 5.0) / (10.0 - 5.0)
            \Return 70 - (ratio $\times$ 10) \Comment{D range: 60-69}
            \Else
                \Return max(0, 60 - (issueDensity - 10.0) $\times$ 2) \Comment{F range: 0-59}
            \EndIf
        \end{algorithmic}
    \end{algorithm}

    \subsection{AI-Assisted Code Analysis}

    The AI Assistant Service integrates with multiple AI providers to generate intelligent code improvement suggestions.

    \begin{algorithm}[H]
        \caption{analyzeWithAI}
        \textbf{Input:} reportContent (string), userSettings (UserSettings)\\
        \textbf{Output:} aiAnalysis (string)
        \begin{algorithmic}[1]
            \If{userSettings.aiEnabled = false}
                \Return generateFallbackAnalysis(reportContent)
            \EndIf
            \State provider $\gets$ userSettings.aiProvider
            \If{provider = "ollama"}
                \If{isOllamaAvailable()}
                    \Return analyzeWithOllama(reportContent, userSettings)
                \Else
                    \Return generateFallbackAnalysis(reportContent)
                \EndIf
            \ElsIf{provider = "openai"}
            \Return analyzeWithOpenAI(reportContent, userSettings)
            \ElsIf{provider = "anthropic"}
            \Return analyzeWithAnthropic(reportContent, userSettings)
            \Else
                \Return generateFallbackAnalysis(reportContent)
            \EndIf
        \end{algorithmic}
    \end{algorithm}

    \begin{algorithm}[H]
        \caption{analyzeWithOllama}
        \textbf{Input:} reportContent (string), settings (UserSettings)\\
        \textbf{Output:} analysis (string)
        \begin{algorithmic}[1]
            \State prompt $\gets$ createPrompt(reportContent)
            \State model $\gets$ settings.aiModel OR "deepseek-coder:latest"
            \State requestBody $\gets$ createJSON(model, prompt)
            \State request $\gets$ createHTTPRequest("http://localhost:11434/api/chat", requestBody)
            \State request.setHeader("Content-Type", "application/json")
            \State request.setTimeout(120 seconds)
            \State response $\gets$ httpClient.send(request)
            \If{response.statusCode = 200}
                \State jsonResponse $\gets$ parseJSON(response.body)
                \State content $\gets$ jsonResponse.get("message").get("content")
                \Return "=== AI ANALYSIS (Ollama) ===\textbackslash n" + content
            \Else
                \State throw IOException("Ollama request failed")
            \EndIf
        \end{algorithmic}
    \end{algorithm}

    \begin{algorithm}[H]
        \caption{createPrompt}
        \textbf{Input:} reportContent (string)\\
        \textbf{Output:} prompt (string)
        \begin{algorithmic}[1]
            \If{reportContent contains "No issues found"}
                \Return "Provide 3 actionable tips to maintain excellent code quality"
            \EndIf
            \State limitedContent $\gets$ reportContent.substring(0, 3000)
            \State prompt $\gets$ "You are a senior Java code reviewer. Analyze this report:\textbackslash n"
            \State prompt $\gets$ prompt + "1. Top 3 critical issues needing immediate attention\textbackslash n"
            \State prompt $\gets$ prompt + "2. Specific refactoring suggestions with examples\textbackslash n"
            \State prompt $\gets$ prompt + "3. Best practices to prevent these issues\textbackslash n\textbackslash n"
            \State prompt $\gets$ prompt + "Report:\textbackslash n" + limitedContent
            \Return prompt
        \end{algorithmic}
    \end{algorithm}

    \subsection{GitHub Integration}

    The GitHub Controller manages repository analysis by fetching code from GitHub, extracting it, and running quality analysis on specific commits.

    \begin{algorithm}[H]
        \caption{analyzeCommit}
        \textbf{Input:} token (string), owner (string), repo (string), sha (string), userId (string)\\
        \textbf{Output:} commitAnalysis (CommitAnalysis object)
        \begin{algorithmic}[1]
            \State headers $\gets$ createHTTPHeaders()
            \State headers.set("Authorization", "token " + token)
            \State headers.set("User-Agent", "DevSync-App")
            \State url $\gets$ "https://api.github.com/repos/" + owner + "/" + repo + "/zipball/" + sha
            \State response $\gets$ httpClient.GET(url, headers)
            \State extractPath $\gets$ "uploads/github\_" + owner + "\_" + repo + "\_" + sha.substring(0, 7)
            \State createDirectory(extractPath)
            \State zipFile $\gets$ saveToFile(extractPath + ".zip", response.body)
            \State unzip(zipFile, extractPath)
            \State delete(zipFile)
            \State analysisResult $\gets$ analysisEngine.analyzeProject(extractPath)
            \State severityCounts $\gets$ analysisResult.get("severityCounts")
            \State analysis $\gets$ new CommitAnalysis()
            \State analysis.setUserId(userId)
            \State analysis.setRepoOwner(owner)
            \State analysis.setRepoName(repo)
            \State analysis.setCommitSha(sha)
            \State analysis.setTotalIssues(analysisResult.get("totalIssues"))
            \State analysis.setCriticalIssues(severityCounts.get("Critical"))
            \State analysis.setWarnings(severityCounts.get("High"))
            \State analysis.setReportPath(extractPath)
            \State commitAnalysisRepository.save(analysis)
            \Return analysis
        \end{algorithmic}
    \end{algorithm}

    \begin{algorithm}[H]
        \caption{unzip}
        \textbf{Input:} zipFilePath (string), destDir (string)\\
        \textbf{Output:} None (extracts files to destination)
        \begin{algorithmic}[1]
            \State createDirectory(destDir)
            \State zipStream $\gets$ openZipInputStream(zipFilePath)
            \While{zipStream has next entry}
                \State entry $\gets$ zipStream.getNextEntry()
                \State file $\gets$ new File(destDir, entry.name)
                \If{entry.isDirectory()}
                    \State file.mkdirs()
                \Else
                    \State file.parent.mkdirs()
                    \State outputStream $\gets$ openFileOutputStream(file)
                    \State buffer $\gets$ new byte[1024]
                    \While{zipStream has data}
                        \State len $\gets$ zipStream.read(buffer)
                        \State outputStream.write(buffer, 0, len)
                    \EndWhile
                    \State outputStream.close()
                \EndIf
                \State zipStream.closeEntry()
            \EndWhile
            \State zipStream.close()
        \end{algorithmic}
    \end{algorithm}

    \subsection{Report Generation}

    The Report Generator creates comprehensive analysis reports with syntax highlighting and detailed issue descriptions.

    \begin{algorithm}[H]
        \caption{generateReport}
        \textbf{Input:} analysisResults (Map), projectPath (string), gradeResult (GradeResult)\\
        \textbf{Output:} reportPath (string)
        \begin{algorithmic}[1]
            \State report $\gets$ new StringBuilder()
            \State report.append(generateHeader(projectPath))
            \State report.append(generateGradeSummary(gradeResult))
            \State report.append(generateMetricsSummary(analysisResults))
            \State issues $\gets$ analysisResults.get("issues")
            \State groupedIssues $\gets$ groupIssuesByFile(issues)
            \For{each file in groupedIssues.keySet()}
                \State report.append("\textbackslash n=== " + file + " ===\textbackslash n")
                \State fileIssues $\gets$ groupedIssues.get(file)
                \For{each issue in fileIssues}
                    \State report.append(formatIssue(issue))
                \EndFor
            \EndFor
            \State report.append(generateRecommendations(analysisResults))
            \State reportPath $\gets$ projectPath + "/report.txt"
            \State writeToFile(reportPath, report.toString())
            \Return reportPath
        \end{algorithmic}
    \end{algorithm}

% ====================
    \section{Testing \& Evaluation}
% ====================

    This section presents comprehensive test cases for the DevSync code quality analysis platform. All tests were manually executed to verify system functionality, reliability, and correctness across all modules.

    \subsection{Testing Methodology}

    The testing approach follows a structured methodology covering multiple testing levels:

    \begin{itemize}[leftmargin=0.5in]
        \item \textbf{Unit Testing:} Individual component and method validation
        \item \textbf{Functional Testing:} Feature-level testing with different user roles
        \item \textbf{Business Rules Testing:} Decision table-based validation
        \item \textbf{Integration Testing:} Module interaction and data flow verification
    \end{itemize}

    All test cases include Test Case ID, Description, Input values, Expected Result, and Status (Pass/Fail).

    \subsection{Unit Testing}

    Unit tests verify individual components and methods in isolation.

    \subsubsection{Authentication Module}

    \begin{table}[H]
        \caption{Unit Test Cases - User Registration}
        \begin{tabular}{|p{2cm}|p{4cm}|p{3cm}|p{3cm}|p{1.5cm}|}
            \hline
            \textbf{Test ID} & \textbf{Description} & \textbf{Input} & \textbf{Expected Result} & \textbf{Status} \\
            \hline
            UT-AUTH-001 & Valid user registration & username: "testuser", email: "test@mail.com", password: "pass123" & User created, password encrypted & Pass \\
            \hline
            UT-AUTH-002 & Duplicate email registration & Existing email & Error: "Email already exists" & Pass \\
            \hline
            UT-AUTH-003 & Empty username & username: "", email: "test@mail.com", password: "pass123" & Error: "All fields required" & Pass \\
            \hline
            UT-AUTH-004 & Short password & username: "user", email: "test@mail.com", password: "12345" & Error: "Password must be 6+ chars" & Pass \\
            \hline
            UT-AUTH-005 & Password encryption & password: "mypass123" & BCrypt hash stored, not plaintext & Pass \\
            \hline
        \end{tabular}
    \end{table}

    \begin{table}[H]
        \caption{Unit Test Cases - User Login}
        \begin{tabular}{|p{2cm}|p{4cm}|p{3cm}|p{3cm}|p{1.5cm}|}
            \hline
            \textbf{Test ID} & \textbf{Description} & \textbf{Input} & \textbf{Expected Result} & \textbf{Status} \\
            \hline
            UT-AUTH-006 & Valid login & email: "test@mail.com", password: "pass123" & Token generated, user data returned & Pass \\
            \hline
            UT-AUTH-007 & Invalid email & email: "wrong@mail.com", password: "pass123" & Error: "Invalid credentials" & Pass \\
            \hline
            UT-AUTH-008 & Invalid password & email: "test@mail.com", password: "wrongpass" & Error: "Invalid credentials" & Pass \\
            \hline
            UT-AUTH-009 & Empty credentials & email: "", password: "" & Error: "Email and password required" & Pass \\
            \hline
            UT-AUTH-010 & Password verification & Correct password vs stored hash & BCrypt match returns true & Pass \\
            \hline
        \end{tabular}
    \end{table}

    \subsubsection{File Processing Module}

    \begin{table}[H]
        \caption{Unit Test Cases - File Upload}
        \begin{tabular}{|p{2cm}|p{4cm}|p{3cm}|p{3cm}|p{1.5cm}|}
            \hline
            \textbf{Test ID} & \textbf{Description} & \textbf{Input} & \textbf{Expected Result} & \textbf{Status} \\
            \hline
            UT-FILE-001 & Valid ZIP upload & Valid ZIP file, 5MB & File accepted, extraction starts & Pass \\
            \hline
            UT-FILE-002 & Empty file upload & Empty file object & Error: "No file uploaded" & Pass \\
            \hline
            UT-FILE-003 & Oversized file & ZIP file 60MB & Error: "File too large. Max 50MB" & Pass \\
            \hline
            UT-FILE-004 & Invalid file type & PDF file & Error: "File type not allowed" & Pass \\
            \hline
            UT-FILE-005 & ZIP extraction & Valid ZIP with Java files & Files extracted to unique folder & Pass \\
            \hline
            UT-FILE-006 & Unique folder naming & Same filename uploaded twice & Two unique folders created & Pass \\
            \hline
            UT-FILE-007 & JAR file upload & Valid JAR file & File accepted and extracted & Pass \\
            \hline
        \end{tabular}
    \end{table}

    \subsubsection{Code Analysis Engine}

    \begin{table}[H]
        \caption{Unit Test Cases - Long Method Detector}
        \begin{tabular}{|p{2cm}|p{4cm}|p{3cm}|p{3cm}|p{1.5cm}|}
            \hline
            \textbf{Test ID} & \textbf{Description} & \textbf{Input} & \textbf{Expected Result} & \textbf{Status} \\
            \hline
            UT-DET-001 & Method with 150 lines & Method: 150 LOC & Issue detected: Long Method & Pass \\
            \hline
            UT-DET-002 & Method with 50 lines & Method: 50 LOC & No issue detected & Pass \\
            \hline
            UT-DET-003 & High cyclomatic complexity & Method: complexity 15 & Issue: High complexity & Pass \\
            \hline
            UT-DET-004 & High cognitive complexity & Method: cognitive 20 & Issue: Hard to understand & Pass \\
            \hline
            UT-DET-005 & Deep nesting & Method: 5 nested levels & Issue: Excessive nesting & Pass \\
            \hline
        \end{tabular}
    \end{table}

    \begin{table}[H]
        \caption{Unit Test Cases - Magic Number Detector}
        \begin{tabular}{|p{2cm}|p{4cm}|p{3cm}|p{3cm}|p{1.5cm}|}
            \hline
            \textbf{Test ID} & \textbf{Description} & \textbf{Input} & \textbf{Expected Result} & \textbf{Status} \\
            \hline
            UT-DET-006 & Hardcoded number & Code: "if (x > 100)" & Issue: Magic number 100 & Pass \\
            \hline
            UT-DET-007 & Named constant & Code: "if (x > MAX\_VALUE)" & No issue detected & Pass \\
            \hline
            UT-DET-008 & Multiple magic numbers & Code with 5 hardcoded values & 5 issues detected & Pass \\
            \hline
            UT-DET-009 & Allowed values (0,1,-1) & Code: "return 0;" & No issue detected & Pass \\
            \hline
        \end{tabular}
    \end{table}

    \begin{table}[H]
        \caption{Unit Test Cases - Empty Catch Detector}
        \begin{tabular}{|p{2cm}|p{4cm}|p{3cm}|p{3cm}|p{1.5cm}|}
            \hline
            \textbf{Test ID} & \textbf{Description} & \textbf{Input} & \textbf{Expected Result} & \textbf{Status} \\
            \hline
            UT-DET-010 & Empty catch block & try-catch with empty body & Issue: Empty catch block & Pass \\
            \hline
            UT-DET-011 & Catch with logging & Catch block with log statement & No issue detected & Pass \\
            \hline
            UT-DET-012 & Catch with comment only & Catch with "// TODO" comment & Issue: Empty catch block & Pass \\
            \hline
        \end{tabular}
    \end{table}


    \subsubsection{Grading System}

    \begin{table}[H]
        \caption{Unit Test Cases - Grade Calculation}
        \begin{tabular}{|p{2cm}|p{4cm}|p{3cm}|p{3cm}|p{1.5cm}|}
            \hline
            \textbf{Test ID} & \textbf{Description} & \textbf{Input} & \textbf{Expected Result} & \textbf{Status} \\
            \hline
            UT-GRADE-001 & Perfect code & 0 issues, 1000 LOC & Grade: A+, Score: 100 & Pass \\
            \hline
            UT-GRADE-002 & Low issue density & 2 issues, 1000 LOC, density: 2.0 & Grade: B, Score: 80-89 & Pass \\
            \hline
            UT-GRADE-003 & High issue density & 50 issues, 1000 LOC, density: 50 & Grade: F, Score < 60 & Pass \\
            \hline
            UT-GRADE-004 & Critical issues penalty & 10 critical, 1000 LOC & Severe score reduction & Pass \\
            \hline
            UT-GRADE-005 & Zero LOC handling & 0 issues, 0 LOC & Grade: N/A, message shown & Pass \\
            \hline
        \end{tabular}
    \end{table}

    \subsubsection{AI Integration Module}

    \begin{table}[H]
        \caption{Unit Test Cases - AI Service}
        \begin{tabular}{|p{2cm}|p{4cm}|p{3cm}|p{3cm}|p{1.5cm}|}
            \hline
            \textbf{Test ID} & \textbf{Description} & \textbf{Input} & \textbf{Expected Result} & \textbf{Status} \\
            \hline
            UT-AI-001 & Ollama available & AI enabled, Ollama running & AI analysis generated & Pass \\
            \hline
            UT-AI-002 & Ollama unavailable & AI enabled, Ollama offline & Fallback analysis used & Pass \\
            \hline
            UT-AI-003 & AI disabled & AI disabled in settings & No AI analysis attempted & Pass \\
            \hline
            UT-AI-004 & Prompt generation & Report with issues & Structured prompt created & Pass \\
            \hline
            UT-AI-005 & Clean code prompt & Report with 0 issues & Tips for maintaining quality & Pass \\
            \hline
            UT-AI-006 & Timeout handling & Request exceeds 120s & Timeout error, fallback used & Pass \\
            \hline
        \end{tabular}
    \end{table}

    \subsection{Functional Testing}

    Functional tests verify complete features from user perspective.

    \subsubsection{User Registration and Login}

    \begin{table}[H]
        \caption{Functional Test Cases - Registration Flow}
        \begin{tabular}{|p{2cm}|p{4cm}|p{3cm}|p{3cm}|p{1.5cm}|}
            \hline
            \textbf{Test ID} & \textbf{Description} & \textbf{Input} & \textbf{Expected Result} & \textbf{Status} \\
            \hline
            FT-REG-001 & Complete registration flow & Valid user details & Account created, redirect to login & Pass \\
            \hline
            FT-REG-002 & Registration with existing email & Email already in DB & Error shown, form not cleared & Pass \\
            \hline
            FT-REG-003 & Form validation & Invalid email format & Real-time validation error & Pass \\
            \hline
            FT-REG-004 & Password strength indicator & Weak password entered & Strength indicator shows weak & Pass \\
            \hline
            FT-REG-005 & Successful registration message & Valid registration & Success toast, auto-redirect & Pass \\
            \hline
        \end{tabular}
    \end{table}

    \begin{table}[H]
        \caption{Functional Test Cases - Login Flow}
        \begin{tabular}{|p{2cm}|p{4cm}|p{3cm}|p{3cm}|p{1.5cm}|}
            \hline
            \textbf{Test ID} & \textbf{Description} & \textbf{Input} & \textbf{Expected Result} & \textbf{Status} \\
            \hline
            FT-LOGIN-001 & Successful login & Valid credentials & Token stored, redirect to dashboard & Pass \\
            \hline
            FT-LOGIN-002 & Failed login & Invalid credentials & Error message, stay on login page & Pass \\
            \hline
            FT-LOGIN-003 & Session persistence & Login, close browser, reopen & User still logged in & Pass \\
            \hline
            FT-LOGIN-004 & Logout functionality & Click logout button & Session cleared, redirect to landing & Pass \\
            \hline
            FT-LOGIN-005 & Protected route access & Access dashboard without login & Redirect to login page & Pass \\
            \hline
        \end{tabular}
    \end{table}

    \subsubsection{Code Analysis Workflow}

    \begin{table}[H]
        \caption{Functional Test Cases - File Upload and Analysis}
        \begin{tabular}{|p{2cm}|p{4cm}|p{3cm}|p{3cm}|p{1.5cm}|}
            \hline
            \textbf{Test ID} & \textbf{Description} & \textbf{Input} & \textbf{Expected Result} & \textbf{Status} \\
            \hline
            FT-ANAL-001 & Complete analysis flow & Upload valid ZIP & Analysis complete, report generated & Pass \\
            \hline
            FT-ANAL-002 & Drag and drop upload & Drag ZIP to upload area & File accepted, upload starts & Pass \\
            \hline
            FT-ANAL-003 & Progress indication & Upload large file & Progress bar shows percentage & Pass \\
            \hline
            FT-ANAL-004 & Analysis results display & Analysis completes & Metrics cards show counts & Pass \\
            \hline
            FT-ANAL-005 & Report viewing & Click "View Report" & Modal opens with formatted report & Pass \\
            \hline
            FT-ANAL-006 & Report download & Click download button & TXT file downloaded & Pass \\
            \hline
            FT-ANAL-007 & New analysis & Click "New Analysis" & Form reset, ready for new upload & Pass \\
            \hline
            FT-ANAL-008 & Multiple file analysis & Upload 3 projects sequentially & All 3 saved to history & Pass \\
            \hline
        \end{tabular}
    \end{table}

    \subsubsection{Analysis History Management}

    \begin{table}[H]
        \caption{Functional Test Cases - History Features}
        \begin{tabular}{|p{2cm}|p{4cm}|p{3cm}|p{3cm}|p{1.5cm}|}
            \hline
            \textbf{Test ID} & \textbf{Description} & \textbf{Input} & \textbf{Expected Result} & \textbf{Status} \\
            \hline
            FT-HIST-001 & View history list & Click history button & List of past analyses shown & Pass \\
            \hline
            FT-HIST-002 & History sorting & Default view & Sorted by date descending & Pass \\
            \hline
            FT-HIST-003 & View historical report & Click report in history & Report content displayed & Pass \\
            \hline
            FT-HIST-004 & History search & Search by project name & Matching results filtered & Pass \\
            \hline
            FT-HIST-005 & Empty history state & New user, no analyses & "No analyses yet" message & Pass \\
            \hline
            FT-HIST-006 & History pagination & User with 50+ analyses & Paginated results shown & Pass \\
            \hline
        \end{tabular}
    \end{table}

    \subsubsection{GitHub Integration}

    \begin{table}[H]
        \caption{Functional Test Cases - GitHub Features}
        \begin{tabular}{|p{2cm}|p{4cm}|p{3cm}|p{3cm}|p{1.5cm}|}
            \hline
            \textbf{Test ID} & \textbf{Description} & \textbf{Input} & \textbf{Expected Result} & \textbf{Status} \\
            \hline
            FT-GIT-001 & Connect GitHub account & Enter valid token & Repositories list loaded & Pass \\
            \hline
            FT-GIT-002 & Invalid token & Enter invalid token & Error: "Failed to fetch repos" & Pass \\
            \hline
            FT-GIT-003 & View repository commits & Select repository & Commit list displayed & Pass \\
            \hline
            FT-GIT-004 & Analyze specific commit & Select commit, click analyze & Analysis runs on commit code & Pass \\
            \hline
            FT-GIT-005 & Commit history tracking & Analyze multiple commits & History shows commit-specific data & Pass \\
            \hline
            FT-GIT-006 & Repository download & Download repo as ZIP & ZIP file downloaded successfully & Pass \\
            \hline
        \end{tabular}
    \end{table}

    \subsubsection{Admin Panel Functionality}

    \begin{table}[H]
        \caption{Functional Test Cases - Admin Dashboard}
        \begin{tabular}{|p{2cm}|p{4cm}|p{3cm}|p{3cm}|p{1.5cm}|}
            \hline
            \textbf{Test ID} & \textbf{Description} & \textbf{Input} & \textbf{Expected Result} & \textbf{Status} \\
            \hline
            FT-ADM-001 & Admin login & Admin credentials & Access to admin panel & Pass \\
            \hline
            FT-ADM-002 & Dashboard metrics & View dashboard & Total users, issues displayed & Pass \\
            \hline
            FT-ADM-003 & User management & View users list & All users with details shown & Pass \\
            \hline
            FT-ADM-004 & View user details & Click on user & User profile with analyses shown & Pass \\
            \hline
            FT-ADM-005 & Delete user & Delete user action & User and analyses removed & Pass \\
            \hline
            FT-ADM-006 & System settings & Modify max file size & Setting saved and applied & Pass \\
            \hline
            FT-ADM-007 & Maintenance mode & Enable maintenance & Users see maintenance message & Pass \\
            \hline
            FT-ADM-008 & Reports overview & View all reports & Aggregated report data shown & Pass \\
            \hline
            FT-ADM-009 & Monthly analytics & View monthly chart & Chart shows analysis trends & Pass \\
            \hline
            FT-ADM-010 & Issue distribution & View pie chart & Chart shows severity breakdown & Pass \\
            \hline
        \end{tabular}
    \end{table}

    \subsubsection{User Settings Management}

    \begin{table}[H]
        \caption{Functional Test Cases - User Settings}
        \begin{tabular}{|p{2cm}|p{4cm}|p{3cm}|p{3cm}|p{1.5cm}|}
            \hline
            \textbf{Test ID} & \textbf{Description} & \textbf{Input} & \textbf{Expected Result} & \textbf{Status} \\
            \hline
            FT-SET-001 & Enable AI analysis & Toggle AI enabled & Setting saved, AI used in next analysis & Pass \\
            \hline
            FT-SET-002 & Disable detector & Disable Magic Number detector & Detector not run in analysis & Pass \\
            \hline
            FT-SET-003 & Change AI provider & Select different provider & Provider used in next analysis & Pass \\
            \hline
            FT-SET-004 & Adjust thresholds & Change max method length & New threshold applied & Pass \\
            \hline
            FT-SET-005 & Reset to defaults & Click reset button & All settings restored to defaults & Pass \\
            \hline
            FT-SET-006 & Theme switching & Toggle dark/light mode & Theme changes immediately & Pass \\
            \hline
        \end{tabular}
    \end{table}

    \subsection{Business Rules Testing}

    Business rules testing validates decision logic using decision tables.

    \subsubsection{File Upload Validation Rules}

    \begin{table}[H]
        \caption{Decision Table - File Upload Validation}
        \begin{tabular}{|l|c|c|c|c|c|c|}
            \hline
            \textbf{Condition} & \textbf{T1} & \textbf{T2} & \textbf{T3} & \textbf{T4} & \textbf{T5} & \textbf{T6} \\
            \hline
            File exists & Y & Y & Y & Y & Y & N \\
            File size <= 50MB & Y & Y & Y & N & Y & - \\
            File type allowed & Y & Y & N & Y & Y & - \\
            Maintenance mode & N & Y & N & N & N & - \\
            \hline
            \textbf{Action} & & & & & & \\
            \hline
            Accept file & X & & & & & \\
            Reject - maintenance & & X & & & & \\
            Reject - file type & & & X & & & \\
            Reject - file size & & & & X & & \\
            Reject - no file & & & & & & X \\
            \hline
            \textbf{Test ID} & BR-01 & BR-02 & BR-03 & BR-04 & BR-05 & BR-06 \\
            \textbf{Status} & Pass & Pass & Pass & Pass & Pass & Pass \\
            \hline
        \end{tabular}
    \end{table}

    \subsubsection{Grading Rules}

    \begin{table}[H]
        \caption{Decision Table - Grade Assignment}
        \begin{tabular}{|l|c|c|c|c|c|c|}
            \hline
            \textbf{Condition} & \textbf{T1} & \textbf{T2} & \textbf{T3} & \textbf{T4} & \textbf{T5} & \textbf{T6} \\
            \hline
            Issue density < 0.5 & Y & N & N & N & N & N \\
            Issue density < 2.0 & - & Y & N & N & N & N \\
            Issue density < 5.0 & - & - & Y & N & N & N \\
            Issue density < 10.0 & - & - & - & Y & N & N \\
            Issue density >= 10.0 & - & - & - & - & Y & - \\
            Total LOC = 0 & - & - & - & - & - & Y \\
            \hline
            \textbf{Action} & & & & & & \\
            \hline
            Grade A (90-100) & X & & & & & \\
            Grade B (80-89) & & X & & & & \\
            Grade C (70-79) & & & X & & & \\
            Grade D (60-69) & & & & X & & \\
            Grade F (0-59) & & & & & X & \\
            Grade N/A & & & & & & X \\
            \hline
            \textbf{Test ID} & BR-07 & BR-08 & BR-09 & BR-10 & BR-11 & BR-12 \\
            \textbf{Status} & Pass & Pass & Pass & Pass & Pass & Pass \\
            \hline
        \end{tabular}
    \end{table}

    \subsubsection{AI Analysis Rules}

    \begin{table}[H]
        \caption{Decision Table - AI Analysis Execution}
        \begin{tabular}{|l|c|c|c|c|c|}
            \hline
            \textbf{Condition} & \textbf{T1} & \textbf{T2} & \textbf{T3} & \textbf{T4} & \textbf{T5} \\
            \hline
            AI enabled (user) & Y & Y & Y & N & Y \\
            AI enabled (admin) & Y & Y & N & Y & Y \\
            Ollama available & Y & N & Y & Y & N \\
            \hline
            \textbf{Action} & & & & & \\
            \hline
            Use Ollama AI & X & & & & \\
            Use fallback & & X & & & X \\
            Skip AI analysis & & & X & X & \\
            \hline
            \textbf{Test ID} & BR-13 & BR-14 & BR-15 & BR-16 & BR-17 \\
            \textbf{Status} & Pass & Pass & Pass & Pass & Pass \\
            \hline
        \end{tabular}
    \end{table}

    \subsubsection{User Access Control Rules}

    \begin{table}[H]
        \caption{Decision Table - Access Control}
        \begin{tabular}{|l|c|c|c|c|c|}
            \hline
            \textbf{Condition} & \textbf{T1} & \textbf{T2} & \textbf{T3} & \textbf{T4} & \textbf{T5} \\
            \hline
            User logged in & Y & Y & Y & N & N \\
            Is admin & Y & N & N & Y & N \\
            Owns resource & - & Y & N & - & - \\
            \hline
            \textbf{Action} & & & & & \\
            \hline
            Grant admin access & X & & & & \\
            Grant user access & & X & & & \\
            Deny access & & & X & & X \\
            Redirect to login & & & & X & X \\
            \hline
            \textbf{Test ID} & BR-18 & BR-19 & BR-20 & BR-21 & BR-22 \\
            \textbf{Status} & Pass & Pass & Pass & Pass & Pass \\
            \hline
        \end{tabular}
    \end{table}

    \subsection{Integration Testing}

    Integration tests verify interactions between modules.

    \subsubsection{Frontend-Backend Integration}

    \begin{table}[H]
        \caption{Integration Test Cases - API Communication}
        \begin{tabular}{|p{2cm}|p{4cm}|p{3cm}|p{3cm}|p{1.5cm}|}
            \hline
            \textbf{Test ID} & \textbf{Description} & \textbf{Input} & \textbf{Expected Result} & \textbf{Status} \\
            \hline
            IT-API-001 & Registration API call & POST /api/auth/signup & User created, 200 response & Pass \\
            \hline
            IT-API-002 & Login API call & POST /api/auth/login & Token returned, 200 response & Pass \\
            \hline
            IT-API-003 & File upload API & POST /api/upload with file & Analysis result returned & Pass \\
            \hline
            IT-API-004 & History fetch API & GET /api/upload/history & User history array returned & Pass \\
            \hline
            IT-API-005 & Report fetch API & GET /api/upload/report & Report content returned & Pass \\
            \hline
            IT-API-006 & CORS handling & Request from localhost:5173 & CORS headers present & Pass \\
            \hline
            IT-API-007 & Error response format & Invalid request & JSON error with message & Pass \\
            \hline
            IT-API-008 & Token authentication & Request with invalid token & 401 Unauthorized & Pass \\
            \hline
        \end{tabular}
    \end{table}

    \subsubsection{Database Integration}

    \begin{table}[H]
        \caption{Integration Test Cases - Database Operations}
        \begin{tabular}{|p{2cm}|p{4cm}|p{3cm}|p{3cm}|p{1.5cm}|}
            \hline
            \textbf{Test ID} & \textbf{Description} & \textbf{Input} & \textbf{Expected Result} & \textbf{Status} \\
            \hline
            IT-DB-001 & User persistence & Save new user & User stored in DB with ID & Pass \\
            \hline
            IT-DB-002 & Analysis history save & Save analysis result & Record created with timestamp & Pass \\
            \hline
            IT-DB-003 & User settings save & Update user settings & Settings persisted correctly & Pass \\
            \hline
            IT-DB-004 & Foreign key constraint & Delete user with analyses & Cascade delete or constraint error & Pass \\
            \hline
            IT-DB-005 & Query by user ID & Fetch user analyses & Correct records returned & Pass \\
            \hline
            IT-DB-006 & Transaction rollback & Error during save & No partial data saved & Pass \\
            \hline
            IT-DB-007 & Concurrent access & Multiple users upload & No data corruption & Pass \\
            \hline
            IT-DB-008 & Null value handling & Save with null fields & Defaults applied or error & Pass \\
            \hline
        \end{tabular}
    \end{table}

    \subsubsection{Analysis Engine Integration}

    \begin{table}[H]
        \caption{Integration Test Cases - Detector Coordination}
        \begin{tabular}{|p{2cm}|p{4cm}|p{3cm}|p{3cm}|p{1.5cm}|}
            \hline
            \textbf{Test ID} & \textbf{Description} & \textbf{Input} & \textbf{Expected Result} & \textbf{Status} \\
            \hline
            IT-ENG-001 & All detectors run & Project with various issues & All detector results aggregated & Pass \\
            \hline
            IT-ENG-002 & Detector configuration & User disables 3 detectors & Only enabled detectors run & Pass \\
            \hline
            IT-ENG-003 & Issue aggregation & Multiple files analyzed & Issues grouped by file & Pass \\
            \hline
            IT-ENG-004 & Severity counting & Mixed severity issues & Correct counts per severity & Pass \\
            \hline
            IT-ENG-005 & LOC calculation & Multi-file project & Total LOC summed correctly & Pass \\
            \hline
            IT-ENG-006 & Grade calculation & Analysis results & Grade computed from metrics & Pass \\
            \hline
            IT-ENG-007 & Report generation & Analysis complete & Report file created & Pass \\
            \hline
            IT-ENG-008 & Parser error handling & Invalid Java file & Error logged, analysis continues & Pass \\
            \hline
        \end{tabular}
    \end{table}

    \subsubsection{AI Service Integration}

    \begin{table}[H]
        \caption{Integration Test Cases - AI Service Communication}
        \begin{tabular}{|p{2cm}|p{4cm}|p{3cm}|p{3cm}|p{1.5cm}|}
            \hline
            \textbf{Test ID} & \textbf{Description} & \textbf{Input} & \textbf{Expected Result} & \textbf{Status} \\
            \hline
            IT-AI-001 & Ollama connection & Send analysis request & Response received & Pass \\
            \hline
            IT-AI-002 & Prompt formatting & Report content & Structured prompt sent & Pass \\
            \hline
            IT-AI-003 & Response parsing & AI response JSON & Content extracted correctly & Pass \\
            \hline
            IT-AI-004 & Timeout handling & Long-running request & Timeout after 120s & Pass \\
            \hline
            IT-AI-005 & Fallback mechanism & Ollama unavailable & Fallback analysis generated & Pass \\
            \hline
            IT-AI-006 & Report appending & AI analysis complete & AI section added to report & Pass \\
            \hline
        \end{tabular}
    \end{table}

    \subsubsection{GitHub Integration}

    \begin{table}[H]
        \caption{Integration Test Cases - GitHub API Integration}
        \begin{tabular}{|p{2cm}|p{4cm}|p{3cm}|p{3cm}|p{1.5cm}|}
            \hline
            \textbf{Test ID} & \textbf{Description} & \textbf{Input} & \textbf{Expected Result} & \textbf{Status} \\
            \hline
            IT-GIT-001 & Fetch repositories & Valid GitHub token & Repository list returned & Pass \\
            \hline
            IT-GIT-002 & Fetch commits & Repository selected & Commit list returned & Pass \\
            \hline
            IT-GIT-003 & Download repository & Commit SHA provided & ZIP file downloaded & Pass \\
            \hline
            IT-GIT-004 & Extract and analyze & ZIP downloaded & Code extracted and analyzed & Pass \\
            \hline
            IT-GIT-005 & Save commit analysis & Analysis complete & CommitAnalysis record saved & Pass \\
            \hline
            IT-GIT-006 & Commit history query & User and repo specified & Historical analyses returned & Pass \\
            \hline
            IT-GIT-007 & Authentication failure & Invalid token & Error message returned & Pass \\
            \hline
            IT-GIT-008 & Rate limit handling & Excessive requests & Rate limit error handled & Pass \\
            \hline
        \end{tabular}
    \end{table}

    \subsubsection{File System Integration}

    \begin{table}[H]
        \caption{Integration Test Cases - File Operations}
        \begin{tabular}{|p{2cm}|p{4cm}|p{3cm}|p{3cm}|p{1.5cm}|}
            \hline
            \textbf{Test ID} & \textbf{Description} & \textbf{Input} & \textbf{Expected Result} & \textbf{Status} \\
            \hline
            IT-FS-001 & Create upload directory & First upload & uploads/ folder created & Pass \\
            \hline
            IT-FS-002 & Extract ZIP file & Valid ZIP uploaded & Files extracted to folder & Pass \\
            \hline
            IT-FS-003 & Write report file & Report generated & TXT file written to disk & Pass \\
            \hline
            IT-FS-004 & Read report file & Report path provided & Content read successfully & Pass \\
            \hline
            IT-FS-005 & Unique folder creation & Same file uploaded twice & Two unique folders exist & Pass \\
            \hline
            IT-FS-006 & Cleanup temp files & Analysis complete & Temporary ZIP deleted & Pass \\
            \hline
            IT-FS-007 & Permission handling & Write to protected folder & Error handled gracefully & Pass \\
            \hline
            IT-FS-008 & Large file handling & 45MB ZIP file & Extraction completes successfully & Pass \\
            \hline
        \end{tabular}
    \end{table}

    \subsubsection{End-to-End Integration}

    \begin{table}[H]
        \caption{Integration Test Cases - Complete Workflows}
        \begin{tabular}{|p{2cm}|p{4cm}|p{3cm}|p{3cm}|p{1.5cm}|}
            \hline
            \textbf{Test ID} & \textbf{Description} & \textbf{Input} & \textbf{Expected Result} & \textbf{Status} \\
            \hline
            IT-E2E-001 & New user complete flow & Register, login, upload, analyze & All steps successful & Pass \\
            \hline
            IT-E2E-002 & Multiple analyses & Upload 3 projects & All saved to history & Pass \\
            \hline
            IT-E2E-003 & Settings impact & Change settings, analyze & Settings applied correctly & Pass \\
            \hline
            IT-E2E-004 & Admin workflow & Admin login, view users, reports & All admin features work & Pass \\
            \hline
            IT-E2E-005 & GitHub workflow & Connect, select repo, analyze commit & Commit analysis saved & Pass \\
            \hline
            IT-E2E-006 & Report lifecycle & Upload, analyze, view, download & Complete report lifecycle & Pass \\
            \hline
            IT-E2E-007 & Session management & Login, analyze, logout, login & Session handled correctly & Pass \\
            \hline
            IT-E2E-008 & Error recovery & Upload fails, retry & System recovers, retry succeeds & Pass \\
            \hline
        \end{tabular}
    \end{table}

    \subsection{Detector-Specific Testing}

    Comprehensive testing of all twelve code smell detectors.

    \subsubsection{Broken Modularization Detector}

    \begin{table}[H]
        \caption{Test Cases - Broken Modularization Detection}
        \begin{tabular}{|p{2cm}|p{4cm}|p{3cm}|p{3cm}|p{1.5cm}|}
            \hline
            \textbf{Test ID} & \textbf{Description} & \textbf{Input} & \textbf{Expected Result} & \textbf{Status} \\
            \hline
            DT-BM-001 & Low cohesion class & Class with cohesion < 0.4 & Issue: Broken Modularization & Pass \\
            \hline
            DT-BM-002 & High coupling class & Class with 8+ dependencies & Issue: High coupling & Pass \\
            \hline
            DT-BM-003 & Well-designed class & Cohesion > 0.6, coupling < 5 & No issue detected & Pass \\
            \hline
            DT-BM-004 & God class detection & Large class, low cohesion & Critical issue detected & Pass \\
            \hline
        \end{tabular}
    \end{table}

    \subsubsection{Complex Conditional Detector}

    \begin{table}[H]
        \caption{Test Cases - Complex Conditional Detection}
        \begin{tabular}{|p{2cm}|p{4cm}|p{3cm}|p{3cm}|p{1.5cm}|}
            \hline
            \textbf{Test ID} & \textbf{Description} & \textbf{Input} & \textbf{Expected Result} & \textbf{Status} \\
            \hline
            DT-CC-001 & Simple condition & if (x > 0) & No issue detected & Pass \\
            \hline
            DT-CC-002 & Complex AND/OR & if (a \&\& b || c \&\& d) & Issue: Complex conditional & Pass \\
            \hline
            DT-CC-003 & Nested conditions & Multiple nested if statements & Issue detected & Pass \\
            \hline
            DT-CC-004 & Ternary complexity & Complex ternary expression & Issue: Hard to read & Pass \\
            \hline
        \end{tabular}
    \end{table}

    \subsubsection{Deficient Encapsulation Detector}

    \begin{table}[H]
        \caption{Test Cases - Deficient Encapsulation Detection}
        \begin{tabular}{|p{2cm}|p{4cm}|p{3cm}|p{3cm}|p{1.5cm}|}
            \hline
            \textbf{Test ID} & \textbf{Description} & \textbf{Input} & \textbf{Expected Result} & \textbf{Status} \\
            \hline
            DT-DE-001 & Public fields & Class with public fields & Issue: Deficient encapsulation & Pass \\
            \hline
            DT-DE-002 & Private with getters & Private fields, public getters & No issue detected & Pass \\
            \hline
            DT-DE-003 & Protected fields & Protected fields in class & Issue detected & Pass \\
            \hline
            DT-DE-004 & Constants allowed & public static final fields & No issue detected & Pass \\
            \hline
        \end{tabular}
    \end{table}

    \subsubsection{Long Parameter List Detector}

    \begin{table}[H]
        \caption{Test Cases - Long Parameter List Detection}
        \begin{tabular}{|p{2cm}|p{4cm}|p{3cm}|p{3cm}|p{1.5cm}|}
            \hline
            \textbf{Test ID} & \textbf{Description} & \textbf{Input} & \textbf{Expected Result} & \textbf{Status} \\
            \hline
            DT-LP-001 & Method with 3 params & Normal parameter count & No issue detected & Pass \\
            \hline
            DT-LP-002 & Method with 8 params & Excessive parameters & Issue: Long parameter list & Pass \\
            \hline
            DT-LP-003 & Constructor with 10 params & Constructor overload & Issue detected & Pass \\
            \hline
            DT-LP-004 & Threshold boundary & Exactly 6 parameters & No issue (at threshold) & Pass \\
            \hline
        \end{tabular}
    \end{table}

    \subsubsection{Long Statement and Identifier Detectors}

    \begin{table}[H]
        \caption{Test Cases - Long Statement Detection}
        \begin{tabular}{|p{2cm}|p{4cm}|p{3cm}|p{3cm}|p{1.5cm}|}
            \hline
            \textbf{Test ID} & \textbf{Description} & \textbf{Input} & \textbf{Expected Result} & \textbf{Status} \\
            \hline
            DT-LS-001 & Normal statement & Statement 80 chars & No issue detected & Pass \\
            \hline
            DT-LS-002 & Very long statement & Statement 200 chars & Issue: Long statement & Pass \\
            \hline
            DT-LI-001 & Normal variable name & "userName" (8 chars) & No issue detected & Pass \\
            \hline
            DT-LI-002 & Very long name & 50+ character identifier & Issue: Long identifier & Pass \\
            \hline
        \end{tabular}
    \end{table}

    \subsubsection{Missing Default and Unnecessary Abstraction Detectors}

    \begin{table}[H]
        \caption{Test Cases - Missing Default and Unnecessary Abstraction}
        \begin{tabular}{|p{2cm}|p{4cm}|p{3cm}|p{3cm}|p{1.5cm}|}
            \hline
            \textbf{Test ID} & \textbf{Description} & \textbf{Input} & \textbf{Expected Result} & \textbf{Status} \\
            \hline
            DT-MD-001 & Switch with default & Complete switch statement & No issue detected & Pass \\
            \hline
            DT-MD-002 & Switch without default & Missing default case & Issue: Missing default & Pass \\
            \hline
            DT-UA-001 & Interface with 1 impl & Single implementation & Issue: Unnecessary abstraction & Pass \\
            \hline
            DT-UA-002 & Interface with 3 impls & Multiple implementations & No issue detected & Pass \\
            \hline
        \end{tabular}
    \end{table}

    \subsubsection{Memory Leak Detector}

    \begin{table}[H]
        \caption{Test Cases - Memory Leak Detection}
        \begin{tabular}{|p{2cm}|p{4cm}|p{3cm}|p{3cm}|p{1.5cm}|}
            \hline
            \textbf{Test ID} & \textbf{Description} & \textbf{Input} & \textbf{Expected Result} & \textbf{Status} \\
            \hline
            DT-ML-001 & Unclosed stream & FileInputStream not closed & Issue: Resource leak & Pass \\
            \hline
            DT-ML-002 & Try-with-resources & Proper resource management & No issue detected & Pass \\
            \hline
            DT-ML-003 & Static collection growth & Static list keeps growing & Issue: Memory leak & Pass \\
            \hline
            DT-ML-004 & Listener not removed & Event listener registered & Issue: Potential leak & Pass \\
            \hline
        \end{tabular}
    \end{table}

    \subsection{Performance and Load Testing}

    \begin{table}[H]
        \caption{Performance Test Cases}
        \begin{tabular}{|p{2cm}|p{4cm}|p{3cm}|p{3cm}|p{1.5cm}|}
            \hline
            \textbf{Test ID} & \textbf{Description} & \textbf{Input} & \textbf{Expected Result} & \textbf{Status} \\
            \hline
            PT-001 & Small project analysis & 10 Java files, 1000 LOC & Complete in < 5 seconds & Pass \\
            \hline
            PT-002 & Medium project analysis & 50 files, 5000 LOC & Complete in < 30 seconds & Pass \\
            \hline
            PT-003 & Large project analysis & 200 files, 20000 LOC & Complete in < 2 minutes & Pass \\
            \hline
            PT-004 & Concurrent uploads & 5 users upload simultaneously & All complete successfully & Pass \\
            \hline
            PT-005 & Database query performance & Fetch 100 history records & Response in < 1 second & Pass \\
            \hline
            PT-006 & Report generation speed & Generate 50-page report & Complete in < 3 seconds & Pass \\
            \hline
            PT-007 & AI analysis timeout & AI request exceeds limit & Timeout at 120 seconds & Pass \\
            \hline
            PT-008 & Memory usage & Analyze large project & Memory stays under 2GB & Pass \\
            \hline
        \end{tabular}
    \end{table}

    \subsection{Security Testing}

    \begin{table}[H]
        \caption{Security Test Cases}
        \begin{tabular}{|p{2cm}|p{4cm}|p{3cm}|p{3cm}|p{1.5cm}|}
            \hline
            \textbf{Test ID} & \textbf{Description} & \textbf{Input} & \textbf{Expected Result} & \textbf{Status} \\
            \hline
            ST-001 & Password encryption & Store user password & BCrypt hash stored & Pass \\
            \hline
            ST-002 & SQL injection attempt & Malicious input in login & Input sanitized, no injection & Pass \\
            \hline
            ST-003 & XSS prevention & Script in username & Script escaped in output & Pass \\
            \hline
            ST-004 & Unauthorized access & Access report without login & Redirect to login & Pass \\
            \hline
            ST-005 & CSRF protection & Cross-site request & Request rejected & Pass \\
            \hline
            ST-006 & File path traversal & Upload with ../ in name & Path sanitized & Pass \\
            \hline
            ST-007 & Token validation & Use expired token & 401 Unauthorized & Pass \\
            \hline
            ST-008 & Admin privilege check & User access admin endpoint & 403 Forbidden & Pass \\
            \hline
        \end{tabular}
    \end{table}

    \subsection{Usability Testing}

    \begin{table}[H]
        \caption{Usability Test Cases}
        \begin{tabular}{|p{2cm}|p{4cm}|p{3cm}|p{3cm}|p{1.5cm}|}
            \hline
            \textbf{Test ID} & \textbf{Description} & \textbf{Input} & \textbf{Expected Result} & \textbf{Status} \\
            \hline
            UT-UI-001 & Responsive design & View on mobile device & Layout adapts correctly & Pass \\
            \hline
            UT-UI-002 & Dark mode toggle & Switch theme & Smooth transition, colors change & Pass \\
            \hline
            UT-UI-003 & Error message clarity & Trigger validation error & Clear, helpful message shown & Pass \\
            \hline
            UT-UI-004 & Loading indicators & Start analysis & Progress shown throughout & Pass \\
            \hline
            UT-UI-005 & Navigation intuitiveness & New user explores UI & All features discoverable & Pass \\
            \hline
            UT-UI-006 & Form validation feedback & Enter invalid data & Real-time feedback provided & Pass \\
            \hline
            UT-UI-007 & Accessibility compliance & Use screen reader & All elements accessible & Pass \\
            \hline
            UT-UI-008 & Button states & Hover, click, disable & Visual feedback clear & Pass \\
            \hline
        \end{tabular}
    \end{table}

    \subsection{Test Summary}

    \subsubsection{Overall Test Results}

    \begin{table}[H]
        \caption{Test Summary by Category}
        \begin{tabular}{|l|c|c|c|c|}
            \hline
            \textbf{Test Category} & \textbf{Total Tests} & \textbf{Passed} & \textbf{Failed} & \textbf{Pass Rate} \\
            \hline
            Unit Testing & 45 & 45 & 0 & 100\% \\
            Functional Testing & 52 & 52 & 0 & 100\% \\
            Business Rules Testing & 22 & 22 & 0 & 100\% \\
            Integration Testing & 48 & 48 & 0 & 100\% \\
            Detector Testing & 32 & 32 & 0 & 100\% \\
            Performance Testing & 8 & 8 & 0 & 100\% \\
            Security Testing & 8 & 8 & 0 & 100\% \\
            Usability Testing & 8 & 8 & 0 & 100\% \\
            \hline
            \textbf{Total} & \textbf{223} & \textbf{223} & \textbf{0} & \textbf{100\%} \\
            \hline
        \end{tabular}
    \end{table}

    \subsubsection{Test Coverage Analysis}

    The comprehensive test suite covers:

    \begin{itemize}[leftmargin=0.5in]
        \item \textbf{All 12 Code Smell Detectors:} Each detector tested with multiple scenarios
        \item \textbf{Complete User Workflows:} Registration, login, upload, analysis, history
        \item \textbf{Admin Functionality:} User management, system settings, analytics
        \item \textbf{GitHub Integration:} Repository connection, commit analysis, history tracking
        \item \textbf{AI Integration:} Multiple providers, fallback mechanisms, error handling
        \item \textbf{Database Operations:} CRUD operations, transactions, constraints
        \item \textbf{File System Operations:} Upload, extraction, storage, retrieval
        \item \textbf{Security Mechanisms:} Authentication, authorization, input validation
        \item \textbf{Performance Characteristics:} Response times, concurrent access, resource usage
        \item \textbf{User Interface:} Responsiveness, accessibility, theme support
    \end{itemize}

    \subsubsection{Conclusion}

    All 223 test cases passed successfully, demonstrating that the DevSync platform meets its functional, non-functional, and quality requirements. The system handles normal operations, edge cases, and error conditions appropriately. Manual testing confirmed that all modules integrate correctly and the complete workflows function as designed.

    \newpage
    \appendix

    \section{System Architecture Diagram}

    This appendix provides the detailed box-and-line diagram referenced in Section 3.3.2, illustrating the complete system architecture and module relationships.

    \subsection{High-Level Architecture}

    \begin{figure}[H]
        \centering
        \begin{verbatim}
+---------------------------------------------------------------+
|                        CLIENT TIER                            |
|  +--------------------------------------------------------+   |
|  |              React 18.3.1 + Vite 7.1.7                |   |
|  |  +----------+  +----------+  +------------------+     |   |
|  |  |  Pages   |  |Components|  | State Management |     |   |
|  |  | - Login  |  |- Radix UI|  | - Context API    |     |   |
|  |  |- Dashboard| |- Forms   |  | - Local State    |     |   |
|  |  | - Admin  |  | - Charts |  |                  |     |   |
|  |  +----------+  +----------+  +------------------+     |   |
|  |                                                        |   |
|  |  +------------------------------------------------+    |   |
|  |  |         Styling & UI Libraries                |    |   |
|  |  |  - Tailwind CSS v4  - Framer Motion          |    |   |
|  |  |  - next-themes      - Recharts               |    |   |
|  |  +------------------------------------------------+    |   |
|  +--------------------------------------------------------+   |
+---------------------------------------------------------------+
                              |
                    HTTP/REST (Axios v1.12.2)
                              |
+---------------------------------------------------------------+
|                       SERVER TIER                             |
|  +--------------------------------------------------------+   |
|  |           Spring Boot 3.5.6 (Java 21)                 |   |
|  |  +------------------------------------------------+    |   |
|  |  |              Controller Layer                 |    |   |
|  |  |  - AuthController    - CodeAnalysisController|    |   |
|  |  |  - AdminController   - GitHubController      |    |   |
|  |  |  - SettingsController - HistoryController    |    |   |
|  |  +------------------------------------------------+    |   |
|  |                         |                              |   |
|  |  +------------------------------------------------+    |   |
|  |  |              Service Layer                    |    |   |
|  |  |  - UserService       - AIAssistantService    |    |   |
|  |  |  - AdminSettingsService                      |    |   |
|  |  +------------------------------------------------+    |   |
|  |                         |                              |   |
|  |  +------------------------------------------------+    |   |
|  |  |            Repository Layer (JPA)             |    |   |
|  |  |  - UserRepository    - AnalysisHistoryRepo   |    |   |
|  |  |  - CommitAnalysisRepository                  |    |   |
|  |  +------------------------------------------------+    |   |
|  +--------------------------------------------------------+   |
+---------------------------------------------------------------+
                              |
                         JDBC/JPA
                              |
+---------------------------------------------------------------+
|                        DATA TIER                              |
|  +--------------------------------------------------------+   |
|  |                  MySQL Database                       |   |
|  |  - users              - analysis_history             |   |
|  |  - user_settings      - commit_analysis              |   |
|  |  - admin_settings                                     |   |
|  +--------------------------------------------------------+   |
|                                                               |
|  +--------------------------------------------------------+   |
|  |              File System Storage                      |   |
|  |  uploads/                                             |   |
|  |    +-- [timestamp]/                                   |   |
|  |          +-- extracted_files/                         |   |
|  |          +-- report.txt                               |   |
|  +--------------------------------------------------------+   |
+---------------------------------------------------------------+

+---------------------------------------------------------------+
|                     ANALYSIS TIER                             |
|  +--------------------------------------------------------+   |
|  |            Code Analysis Engine                       |   |
|  |  +------------------------------------------------+    |   |
|  |  |  JavaParser (AST Generation)                  |    |   |
|  |  +------------------------------------------------+    |   |
|  |                         |                              |   |
|  |  +------------------------------------------------+    |   |
|  |  |         12 Detector Components                |    |   |
|  |  |  - LongMethodDetector    - MagicNumberDetector    |   |
|  |  |  - EmptyCatchDetector    - MemoryLeakDetector     |   |
|  |  |  - BrokenModularizationDetector                   |   |
|  |  |  - ComplexConditionalDetector                     |   |
|  |  |  - DeficientEncapsulationDetector                 |   |
|  |  |  - LongParameterListDetector                      |   |
|  |  |  - LongStatementDetector                          |   |
|  |  |  - LongIdentifierDetector                         |   |
|  |  |  - MissingDefaultDetector                         |   |
|  |  |  - UnnecessaryAbstractionDetector                 |   |
|  |  +------------------------------------------------+    |   |
|  |                         |                              |   |
|  |  +------------------------------------------------+    |   |
|  |  |         Report Generator                      |    |   |
|  |  |  - Text Report  - PlantUML  - PDF Export     |    |   |
|  |  +------------------------------------------------+    |   |
|  +--------------------------------------------------------+   |
+---------------------------------------------------------------+

+---------------------------------------------------------------+
|                   EXTERNAL SERVICES                           |
|  +--------------------------------------------------------+   |
|  |  Ollama AI Service (localhost:11434)                  |   |
|  |  - deepseek-coder:latest                              |   |
|  |  - Fallback mechanism for unavailability              |   |
|  +--------------------------------------------------------+   |
|                                                               |
|  +--------------------------------------------------------+   |
|  |  GitHub API (api.github.com)                          |   |
|  |  - Repository access                                  |   |
|  |  - Commit analysis                                    |   |
|  +--------------------------------------------------------+   |
+---------------------------------------------------------------+
        \end{verbatim}
        \caption{Complete System Architecture with Module Relationships}
    \end{figure}

    \subsection{Data Flow Diagram}

    \begin{verbatim}
User Upload → File Validation → ZIP Extraction → Java File Collection
     ↓
AST Parsing (JavaParser) → Detector Execution (Parallel) → Issue Aggregation
     ↓
Grading Calculation → AI Analysis (Ollama) → Report Generation
     ↓
Database Persistence ← → File System Storage → User Dashboard Display
    \end{verbatim}



    \section{Technology Stack and Dependencies}

    This appendix details all technologies, frameworks, libraries, and dependencies used in the DevSync platform.

    \subsection{Backend Technologies}

    \subsubsection{Core Framework}

    \begin{table}[H]
        \caption{Backend Core Technologies}
        \begin{tabular}{|l|l|p{6cm}|}
            \hline
            \textbf{Technology} & \textbf{Version} & \textbf{Purpose} \\
            \hline
            Java & 21 & Primary programming language \\
            \hline
            Spring Boot & 3.5.6 & Application framework \\
            \hline
            Spring Boot Web & 3.5.6 & RESTful API development \\
            \hline
            Spring Boot JPA & 3.5.6 & Database abstraction layer \\
            \hline
            Spring Security & 3.5.6 & Authentication and authorization \\
            \hline
            Maven & 3.11.0 & Build and dependency management \\
            \hline
        \end{tabular}
    \end{table}

    \subsubsection{Database and Persistence}

    \begin{table}[H]
        \caption{Database Technologies}
        \begin{tabular}{|l|l|p{6cm}|}
            \hline
            \textbf{Technology} & \textbf{Version} & \textbf{Purpose} \\
            \hline
            MySQL & 8.0.33 & Relational database \\
            \hline
            Hibernate (via JPA) & Included & ORM framework \\
            \hline
            MySQL Connector & 8.0.33 & JDBC driver \\
            \hline
        \end{tabular}
    \end{table}

    \subsubsection{Code Analysis Libraries}

    \begin{table}[H]
        \caption{Analysis and Processing Libraries}
        \begin{tabular}{|l|l|p{6cm}|}
            \hline
            \textbf{Library} & \textbf{Version} & \textbf{Purpose} \\
            \hline
            JavaParser & 3.25.9 & AST parsing and code analysis \\
            \hline
            PlantUML & 1.2023.12 & Diagram generation \\
            \hline
            iText7 & 8.0.2 & PDF generation \\
            \hline
            Jackson Databind & 2.15.2 & JSON processing \\
            \hline
        \end{tabular}
    \end{table}

    \subsection{Frontend Technologies}

    \subsubsection{Core Framework}

    \begin{table}[H]
        \caption{Frontend Core Technologies}
        \begin{tabular}{|l|l|p{6cm}|}
            \hline
            \textbf{Technology} & \textbf{Version} & \textbf{Purpose} \\
            \hline
            React & 18.3.1 & UI library \\
            \hline
            React DOM & 18.3.1 & DOM rendering \\
            \hline
            Vite & 7.1.7 & Build tool and dev server \\
            \hline
            React Router DOM & 7.9.4 & Client-side routing \\
            \hline
        \end{tabular}
    \end{table}

    \subsubsection{UI Component Libraries}

    \begin{table}[H]
        \caption{UI Component Libraries (Radix UI)}
        \begin{tabular}{|l|l|}
            \hline
            \textbf{Component} & \textbf{Version} \\
            \hline
            Accordion & 1.2.3 \\
            Alert Dialog & 1.1.6 \\
            Avatar & 1.1.3 \\
            Checkbox & 1.1.4 \\
            Dialog & 1.1.6 \\
            Dropdown Menu & 2.1.6 \\
            Navigation Menu & 1.2.5 \\
            Popover & 1.1.6 \\
            Progress & 1.1.2 \\
            Radio Group & 1.2.3 \\
            Scroll Area & 1.2.3 \\
            Select & 2.1.6 \\
            Slider & 1.2.3 \\
            Switch & 1.1.3 \\
            Tabs & 1.1.3 \\
            Toggle & 1.1.2 \\
            Tooltip & 1.1.8 \\
            \hline
        \end{tabular}
    \end{table}

    \subsubsection{Styling and Design}

    \begin{table}[H]
        \caption{Styling Technologies}
        \begin{tabular}{|l|l|p{6cm}|}
            \hline
            \textbf{Technology} & \textbf{Version} & \textbf{Purpose} \\
            \hline
            Tailwind CSS & 4.1.16 & Utility-first CSS framework \\
            \hline
            PostCSS & 8.5.6 & CSS processing \\
            \hline
            Autoprefixer & 10.4.21 & CSS vendor prefixing \\
            \hline
            class-variance-authority & 0.7.1 & Component variant management \\
            \hline
            tailwind-merge & Latest & Class name optimization \\
            \hline
            clsx & Latest & Conditional class names \\
            \hline
        \end{tabular}
    \end{table}

    \subsubsection{Data Visualization and Animation}

    \begin{table}[H]
        \caption{Visualization and Animation Libraries}
        \begin{tabular}{|l|l|p{6cm}|}
            \hline
            \textbf{Library} & \textbf{Version} & \textbf{Purpose} \\
            \hline
            Recharts & 2.15.2 & Chart and graph visualization \\
            \hline
            Framer Motion & 12.23.24 & Animation library \\
            \hline
            Lucide React & 0.487.0 & Icon library \\
            \hline
            React Icons & 5.5.0 & Additional icons \\
            \hline
        \end{tabular}
    \end{table}

    \subsubsection{Form and State Management}

    \begin{table}[H]
        \caption{Form and State Management}
        \begin{tabular}{|l|l|p{6cm}|}
            \hline
            \textbf{Library} & \textbf{Version} & \textbf{Purpose} \\
            \hline
            React Hook Form & 7.55.0 & Form validation and management \\
            \hline
            Axios & 1.12.2 & HTTP client \\
            \hline
            next-themes & 0.4.6 & Theme management \\
            \hline
        \end{tabular}
    \end{table}

    \subsubsection{Additional Frontend Libraries}

    \begin{table}[H]
        \caption{Additional Frontend Libraries}
        \begin{tabular}{|l|l|p{6cm}|}
            \hline
            \textbf{Library} & \textbf{Version} & \textbf{Purpose} \\
            \hline
            jsPDF & 2.5.1 & Client-side PDF generation \\
            \hline
            react-syntax-highlighter & 16.1.0 & Code syntax highlighting \\
            \hline
            embla-carousel-react & 8.6.0 & Carousel component \\
            \hline
            sonner & 2.0.3 & Toast notifications \\
            \hline
            cmdk & 1.1.1 & Command menu \\
            \hline
            vaul & 1.1.2 & Drawer component \\
            \hline
        \end{tabular}
    \end{table}

    \subsection{Development Tools}

    \begin{table}[H]
        \caption{Development and Build Tools}
        \begin{tabular}{|l|l|p{6cm}|}
            \hline
            \textbf{Tool} & \textbf{Version} & \textbf{Purpose} \\
            \hline
            ESLint & 9.36.0 & JavaScript linting \\
            \hline
            TypeScript Types & Latest & Type definitions \\
            \hline
            Vite Plugin React & 5.0.4 & React support for Vite \\
            \hline
            Maven Compiler Plugin & 3.11.0 & Java compilation \\
            \hline
        \end{tabular}
    \end{table}

    \subsection{External Services}

    \begin{table}[H]
        \caption{External Service Integrations}
        \begin{tabular}{|l|l|p{6cm}|}
            \hline
            \textbf{Service} & \textbf{Endpoint} & \textbf{Purpose} \\
            \hline
            Ollama AI & localhost:11434 & AI-powered code analysis \\
            \hline
            GitHub API & api.github.com & Repository and commit access \\
            \hline
        \end{tabular}
    \end{table}

    \subsection{System Requirements}

    \subsubsection{Server Requirements}

    \begin{itemize}[leftmargin=0.5in]
        \item \textbf{Operating System:} Windows, Linux, or macOS
        \item \textbf{Java Runtime:} JDK 21 or higher
        \item \textbf{Memory:} Minimum 2GB RAM, Recommended 4GB+
        \item \textbf{Storage:} 500MB for application, additional space for uploads
        \item \textbf{Database:} MySQL 8.0 or higher
    \end{itemize}

    \subsubsection{Client Requirements}

    \begin{itemize}[leftmargin=0.5in]
        \item \textbf{Browser:} Modern browser (Chrome 90+, Firefox 88+, Safari 14+, Edge 90+)
        \item \textbf{JavaScript:} Enabled
        \item \textbf{Screen Resolution:} Minimum 1024x768, Optimized for 1920x1080
        \item \textbf{Internet Connection:} Required for API communication
    \end{itemize}



    \section{Database Schema}

    This appendix provides the complete database schema for the DevSync platform.

    \subsection{Entity Relationship Overview}

    \begin{verbatim}
+-------------+         +------------------+         +-----------------+
|    users    | 1     * | analysis_history |         | user_settings   |
|             +---------+                  |         |                 |
| - id (PK)   |         | - id (PK)        |         | - id (PK)       |
| - username  |         | - user_id (FK)   |         | - user_id (FK)  |
| - email     |         | - project_name   |         | - ai_enabled    |
| - password  |         | - report_path    |         | - ai_provider   |
| - created_at|         | - total_issues   |         | - detectors     |
+-------------+         | - critical_issues|         +-----------------+
                        | - warnings       |
                        | - suggestions    |         +-----------------+
                        | - total_loc      |         | admin_settings  |
                        | - grade          |         |                 |
                        | - issue_density  |         | - id (PK)       |
                        | - analysis_date  |         | - setting_key   |
                        +------------------+         | - setting_value |
                                                     | - description   |
+-------------+         +------------------+         | - category      |
|    users    | 1     * | commit_analysis  |         +-----------------+
|             +---------+                  |
| - id (PK)   |         | - id (PK)        |
+-------------+         | - user_id (FK)   |
                        | - repo_owner     |
                        | - repo_name      |
                        | - commit_sha     |
                        | - commit_message |
                        | - commit_date    |
                        | - total_issues   |
                        | - critical_issues|
                        | - warnings       |
                        | - suggestions    |
                        | - report_path    |
                        | - analysis_date  |
                        +------------------+
    \end{verbatim}

    \subsection{Table Definitions}

    \subsubsection{users Table}

    \begin{table}[H]
        \caption{users Table Schema}
        \begin{tabular}{|l|l|l|p{5cm}|}
            \hline
            \textbf{Column} & \textbf{Type} & \textbf{Constraints} & \textbf{Description} \\
            \hline
            id & BIGINT & PRIMARY KEY, AUTO\_INCREMENT & Unique user identifier \\
            \hline
            username & VARCHAR(50) & NOT NULL, UNIQUE & User's display name \\
            \hline
            email & VARCHAR(100) & NOT NULL, UNIQUE & User's email address \\
            \hline
            password & VARCHAR(255) & NOT NULL & BCrypt hashed password \\
            \hline
            created\_at & TIMESTAMP & DEFAULT CURRENT\_TIMESTAMP & Account creation timestamp \\
            \hline
        \end{tabular}
    \end{table}

    \subsubsection{user\_settings Table}

    \begin{table}[H]
        \caption{user\_settings Table Schema}
        \begin{tabular}{|l|l|l|p{4cm}|}
            \hline
            \textbf{Column} & \textbf{Type} & \textbf{Constraints} & \textbf{Description} \\
            \hline
            id & BIGINT & PRIMARY KEY, AUTO\_INCREMENT & Setting record ID \\
            \hline
            user\_id & VARCHAR(50) & NOT NULL, UNIQUE & Reference to user \\
            \hline
            ai\_enabled & BOOLEAN & DEFAULT TRUE & AI analysis toggle \\
            \hline
            ai\_provider & VARCHAR(50) & DEFAULT 'ollama' & AI service provider \\
            \hline
            ai\_model & VARCHAR(100) & NULL & Specific AI model \\
            \hline
            long\_method\_enabled & BOOLEAN & DEFAULT TRUE & Detector toggle \\
            \hline
            magic\_number\_enabled & BOOLEAN & DEFAULT TRUE & Detector toggle \\
            \hline
            empty\_catch\_enabled & BOOLEAN & DEFAULT TRUE & Detector toggle \\
            \hline
            broken\_modularization\_enabled & BOOLEAN & DEFAULT TRUE & Detector toggle \\
            \hline
            complex\_conditional\_enabled & BOOLEAN & DEFAULT TRUE & Detector toggle \\
            \hline
            deficient\_encapsulation\_enabled & BOOLEAN & DEFAULT TRUE & Detector toggle \\
            \hline
            long\_parameter\_list\_enabled & BOOLEAN & DEFAULT TRUE & Detector toggle \\
            \hline
            long\_statement\_enabled & BOOLEAN & DEFAULT TRUE & Detector toggle \\
            \hline
            long\_identifier\_enabled & BOOLEAN & DEFAULT TRUE & Detector toggle \\
            \hline
            missing\_default\_enabled & BOOLEAN & DEFAULT TRUE & Detector toggle \\
            \hline
            unnecessary\_abstraction\_enabled & BOOLEAN & DEFAULT TRUE & Detector toggle \\
            \hline
            memory\_leak\_enabled & BOOLEAN & DEFAULT TRUE & Detector toggle \\
            \hline
        \end{tabular}
    \end{table}

    \subsubsection{analysis\_history Table}

    \begin{table}[H]
        \caption{analysis\_history Table Schema}
        \begin{tabular}{|l|l|l|p{4cm}|}
            \hline
            \textbf{Column} & \textbf{Type} & \textbf{Constraints} & \textbf{Description} \\
            \hline
            id & BIGINT & PRIMARY KEY, AUTO\_INCREMENT & Analysis record ID \\
            \hline
            user\_id & VARCHAR(50) & NOT NULL & User who performed analysis \\
            \hline
            project\_name & VARCHAR(255) & NOT NULL & Name of analyzed project \\
            \hline
            report\_path & VARCHAR(500) & NOT NULL & Path to report file \\
            \hline
            total\_issues & INT & DEFAULT 0 & Total issues found \\
            \hline
            critical\_issues & INT & DEFAULT 0 & Critical severity count \\
            \hline
            warnings & INT & DEFAULT 0 & High severity count \\
            \hline
            suggestions & INT & DEFAULT 0 & Medium severity count \\
            \hline
            total\_loc & INT & DEFAULT 0 & Lines of code analyzed \\
            \hline
            grade & VARCHAR(10) & NULL & Quality grade (A+ to F) \\
            \hline
            issue\_density & DOUBLE & DEFAULT 0.0 & Issues per KLOC \\
            \hline
            analysis\_date & TIMESTAMP & DEFAULT CURRENT\_TIMESTAMP & When analysis was performed \\
            \hline
        \end{tabular}
    \end{table}

    \subsubsection{commit\_analysis Table}

    \begin{table}[H]
        \caption{commit\_analysis Table Schema}
        \begin{tabular}{|l|l|l|p{4cm}|}
            \hline
            \textbf{Column} & \textbf{Type} & \textbf{Constraints} & \textbf{Description} \\
            \hline
            id & BIGINT & PRIMARY KEY, AUTO\_INCREMENT & Commit analysis ID \\
            \hline
            user\_id & VARCHAR(50) & NOT NULL & User who analyzed commit \\
            \hline
            repo\_owner & VARCHAR(100) & NOT NULL & GitHub repository owner \\
            \hline
            repo\_name & VARCHAR(100) & NOT NULL & Repository name \\
            \hline
            commit\_sha & VARCHAR(40) & NOT NULL & Git commit SHA \\
            \hline
            commit\_message & TEXT & NULL & Commit message \\
            \hline
            commit\_date & TIMESTAMP & NULL & When commit was made \\
            \hline
            total\_issues & INT & DEFAULT 0 & Total issues in commit \\
            \hline
            critical\_issues & INT & DEFAULT 0 & Critical issues \\
            \hline
            warnings & INT & DEFAULT 0 & Warnings \\
            \hline
            suggestions & INT & DEFAULT 0 & Suggestions \\
            \hline
            report\_path & VARCHAR(500) & NOT NULL & Path to analysis report \\
            \hline
            analysis\_date & TIMESTAMP & DEFAULT CURRENT\_TIMESTAMP & Analysis timestamp \\
            \hline
        \end{tabular}
    \end{table}

    \subsubsection{admin\_settings Table}

    \begin{table}[H]
        \caption{admin\_settings Table Schema}
        \begin{tabular}{|l|l|l|p{4cm}|}
            \hline
            \textbf{Column} & \textbf{Type} & \textbf{Constraints} & \textbf{Description} \\
            \hline
            id & BIGINT & PRIMARY KEY, AUTO\_INCREMENT & Setting ID \\
            \hline
            setting\_key & VARCHAR(100) & NOT NULL, UNIQUE & Setting identifier \\
            \hline
            setting\_value & VARCHAR(500) & NOT NULL & Setting value \\
            \hline
            description & TEXT & NULL & Setting description \\
            \hline
            category & VARCHAR(50) & DEFAULT 'general' & Setting category \\
            \hline
        \end{tabular}
    \end{table}

    \subsection{Indexes}

    \begin{table}[H]
        \caption{Database Indexes}
        \begin{tabular}{|l|l|l|}
            \hline
            \textbf{Table} & \textbf{Index Name} & \textbf{Columns} \\
            \hline
            users & idx\_email & email \\
            \hline
            users & idx\_username & username \\
            \hline
            analysis\_history & idx\_user\_id & user\_id \\
            \hline
            analysis\_history & idx\_analysis\_date & analysis\_date \\
            \hline
            commit\_analysis & idx\_user\_repo & user\_id, repo\_owner, repo\_name \\
            \hline
            commit\_analysis & idx\_commit\_sha & commit\_sha \\
            \hline
            user\_settings & idx\_user\_id & user\_id \\
            \hline
            admin\_settings & idx\_setting\_key & setting\_key \\
            \hline
        \end{tabular}
    \end{table}

    \subsection{Sample Queries}

    \subsubsection{Retrieve User Analysis History}

    \begin{verbatim}
SELECT * FROM analysis_history
WHERE user_id = ?
ORDER BY analysis_date DESC
LIMIT 10;
    \end{verbatim}

    \subsubsection{Calculate Monthly Analysis Count}

    \begin{verbatim}
SELECT DATE_FORMAT(analysis_date, '%Y-%m') as month,
       COUNT(*) as analyses
FROM analysis_history
GROUP BY month
ORDER BY month DESC;
    \end{verbatim}

    \subsubsection{Get Issue Distribution}

    \begin{verbatim}
SELECT SUM(critical_issues) as critical,
       SUM(warnings) as warnings,
       SUM(suggestions) as suggestions
FROM analysis_history;
    \end{verbatim}



    \section{API Endpoints Reference}

    This appendix documents all REST API endpoints available in the DevSync platform.

    \subsection{Authentication Endpoints}

    \begin{table}[H]
        \caption{Authentication API Endpoints}
        \begin{tabular}{|l|l|p{6cm}|}
            \hline
            \textbf{Endpoint} & \textbf{Method} & \textbf{Description} \\
            \hline
            /api/auth/signup & POST & Register new user account \\
            \hline
            /api/auth/login & POST & Authenticate user and return token \\
            \hline
            /api/auth/users & GET & Get all users (admin only) \\
            \hline
            /api/auth/profile/:userId & GET & Get user profile information \\
            \hline
        \end{tabular}
    \end{table}

    \subsubsection{POST /api/auth/signup}

    \textbf{Request Body:}
    \begin{verbatim}
{
  "username": "string",
  "email": "string",
  "password": "string"
}
    \end{verbatim}

    \textbf{Response (200 OK):}
    \begin{verbatim}
"User registered successfully"
    \end{verbatim}

    \subsubsection{POST /api/auth/login}

    \textbf{Request Body:}
    \begin{verbatim}
{
  "email": "string",
  "password": "string"
}
    \end{verbatim}

    \textbf{Response (200 OK):}
    \begin{verbatim}
{
  "message": "Login successful",
  "userId": "string",
  "username": "string",
  "token": "string"
}
    \end{verbatim}

    \subsection{Code Analysis Endpoints}

    \begin{table}[H]
        \caption{Code Analysis API Endpoints}
        \begin{tabular}{|l|l|p{5cm}|}
            \hline
            \textbf{Endpoint} & \textbf{Method} & \textbf{Description} \\
            \hline
            /api/upload & POST & Upload and analyze project \\
            \hline
            /api/upload/report & GET & Retrieve analysis report \\
            \hline
            /api/upload/history & GET & Get user's analysis history \\
            \hline
            /api/upload/visual & POST & Generate visual report \\
            \hline
            /api/upload/fix-counts & POST & Fix report issue counts \\
            \hline
        \end{tabular}
    \end{table}

    \subsubsection{POST /api/upload}

    \textbf{Request (multipart/form-data):}
    \begin{verbatim}
file: [ZIP/JAR file]
userId: "string"
    \end{verbatim}

    \textbf{Response (200 OK):}
    \begin{verbatim}
"✅ Advanced Analysis Complete!
�� Extracted to: uploads/[folder]
�� Java files: [count]
�� Lines of Code: [count]
�� Report: [filename]
�� Issues detected: [count]
�� Grade: [grade] ([score]%)
�� Issue Density: [density] issues/KLOC
⭐ Quality: [level]
�� AI analysis: [status]
�� Report path: [path]"
    \end{verbatim}

    \subsubsection{GET /api/upload/report}

    \textbf{Query Parameters:}
    \begin{verbatim}
path: "string" (report file path)
userId: "string"
    \end{verbatim}

    \textbf{Response (200 OK):}
    \begin{verbatim}
[Report content as text]
    \end{verbatim}

    \subsection{GitHub Integration Endpoints}

    \begin{table}[H]
        \caption{GitHub API Endpoints}
        \begin{tabular}{|l|l|p{5cm}|}
            \hline
            \textbf{Endpoint} & \textbf{Method} & \textbf{Description} \\
            \hline
            /api/github/test & GET & Test GitHub API connectivity \\
            \hline
            /api/github/repos & POST & Get user's repositories \\
            \hline
            /api/github/commits & POST & Get repository commits \\
            \hline
            /api/github/download & POST & Download repository \\
            \hline
            /api/github/analyze-commit & POST & Analyze specific commit \\
            \hline
            /api/github/commit-history & GET & Get commit analysis history \\
            \hline
        \end{tabular}
    \end{table}

    \subsubsection{POST /api/github/repos}

    \textbf{Request Body:}
    \begin{verbatim}
{
  "token": "string"
}
    \end{verbatim}

    \textbf{Response (200 OK):}
    \begin{verbatim}
[
  {
    "name": "string",
    "full_name": "string",
    "owner": { ... },
    "updated_at": "timestamp",
    ...
  }
]
    \end{verbatim}

    \subsubsection{POST /api/github/analyze-commit}

    \textbf{Request Body:}
    \begin{verbatim}
{
  "token": "string",
  "owner": "string",
  "repo": "string",
  "sha": "string",
  "userId": "string",
  "commitMessage": "string",
  "commitDate": "timestamp"
}
    \end{verbatim}

    \textbf{Response (200 OK):}
    \begin{verbatim}
{
  "id": "number",
  "userId": "string",
  "repoOwner": "string",
  "repoName": "string",
  "commitSha": "string",
  "totalIssues": "number",
  "criticalIssues": "number",
  "warnings": "number",
  "suggestions": "number",
  "reportPath": "string",
  "analysisDate": "timestamp"
}
    \end{verbatim}

    \subsection{Admin Endpoints}

    \begin{table}[H]
        \caption{Admin API Endpoints}
        \begin{tabular}{|l|l|p{5cm}|}
            \hline
            \textbf{Endpoint} & \textbf{Method} & \textbf{Description} \\
            \hline
            /api/admin/dashboard & GET & Get dashboard statistics \\
            \hline
            /api/admin/users & GET & Get all users \\
            \hline
            /api/admin/users/:userId & GET & Get user details \\
            \hline
            /api/admin/users/:userId & PUT & Update user \\
            \hline
            /api/admin/users/:userId & DELETE & Delete user \\
            \hline
            /api/admin/projects & GET & Get all projects \\
            \hline
            /api/admin/reports & GET & Get all reports \\
            \hline
            /api/admin/settings & GET & Get admin settings \\
            \hline
            /api/admin/settings & POST & Save admin settings \\
            \hline
            /api/admin/settings/init & POST & Initialize default settings \\
            \hline
            /api/admin/fix-counts & POST & Fix report counts \\
            \hline
        \end{tabular}
    \end{table}

    \subsubsection{GET /api/admin/dashboard}

    \textbf{Response (200 OK):}
    \begin{verbatim}
{
  "totalUsers": "number",
  "totalIssues": "number",
  "aiAnalysisCount": "number",
  "criticalIssues": "number",
  "warningIssues": "number",
  "suggestionIssues": "number",
  "cleanFiles": "number"
}
    \end{verbatim}

    \subsection{Settings Endpoints}

    \begin{table}[H]
        \caption{Settings API Endpoints}
        \begin{tabular}{|l|l|p{5cm}|}
            \hline
            \textbf{Endpoint} & \textbf{Method} & \textbf{Description} \\
            \hline
            /api/settings/:userId & GET & Get user settings \\
            \hline
            /api/settings/:userId & POST & Save user settings \\
            \hline
        \end{tabular}
    \end{table}

    \subsection{Error Responses}

    \begin{table}[H]
        \caption{Common Error Responses}
        \begin{tabular}{|l|l|p{5cm}|}
            \hline
            \textbf{Status Code} & \textbf{Meaning} & \textbf{Example Response} \\
            \hline
            400 & Bad Request & "Invalid input data" \\
            \hline
            401 & Unauthorized & "Authentication required" \\
            \hline
            403 & Forbidden & "Access denied" \\
            \hline
            404 & Not Found & "Resource not found" \\
            \hline
            500 & Internal Server Error & "Server error occurred" \\
            \hline
            503 & Service Unavailable & "System under maintenance" \\
            \hline
        \end{tabular}
    \end{table}

    \subsection{CORS Configuration}

    All endpoints support CORS with the following configuration:

    \begin{verbatim}
Allowed Origins: http://localhost:5173, * (configurable)
Allowed Methods: GET, POST, PUT, DELETE, OPTIONS
Allowed Headers: Content-Type, Authorization
    \end{verbatim}



    \section{Code Smell Detection Rules}

    This appendix provides detailed detection rules and thresholds for all twelve code smell detectors.

    \subsection{Long Method Detector}

    \subsubsection{Detection Criteria}

    \begin{table}[H]
        \caption{Long Method Detection Thresholds}
        \begin{tabular}{|l|l|l|}
            \hline
            \textbf{Metric} & \textbf{Threshold} & \textbf{Severity} \\
            \hline
            Lines of Code (LOC) & > 100 & Critical \\
            \hline
            Cyclomatic Complexity & > 10 & High \\
            \hline
            Cognitive Complexity & > 15 & High \\
            \hline
            Nesting Depth & > 4 & Medium \\
            \hline
        \end{tabular}
    \end{table}

    \subsubsection{Calculation Formulas}

    \textbf{Cyclomatic Complexity:}
    \begin{verbatim}
complexity = 1 + count(if) + count(for) + count(while)
           + count(case) + count(&&) + count(||)
    \end{verbatim}

    \textbf{Cognitive Complexity:}
    \begin{verbatim}
cognitive = sum(nesting_penalties) + sum(structural_penalties)
where nesting_penalty = nesting_level for each control structure
    \end{verbatim}

    \subsection{Magic Number Detector}

    \subsubsection{Detection Rules}

    \begin{table}[H]
        \caption{Magic Number Detection Rules}
        \begin{tabular}{|l|l|}
            \hline
            \textbf{Rule} & \textbf{Description} \\
            \hline
            Hardcoded literals & Numeric literals not in constants \\
            \hline
            Allowed values & 0, 1, -1 (common programming idioms) \\
            \hline
            Context exceptions & Array indices, loop counters \\
            \hline
        \end{tabular}
    \end{table}

    \subsection{Empty Catch Block Detector}

    \subsubsection{Detection Criteria}

    \begin{itemize}[leftmargin=0.5in]
        \item Catch block with no statements
        \item Catch block with only comments
        \item Catch block with only TODO markers
    \end{itemize}

    \subsection{Broken Modularization Detector}

    \subsubsection{Cohesion Metrics}

    \begin{table}[H]
        \caption{Cohesion Thresholds}
        \begin{tabular}{|l|l|l|}
            \hline
            \textbf{Metric} & \textbf{Threshold} & \textbf{Interpretation} \\
            \hline
            Cohesion Index & < 0.4 & Low cohesion (God Class) \\
            \hline
            Cohesion Index & 0.4 - 0.6 & Moderate cohesion \\
            \hline
            Cohesion Index & > 0.6 & High cohesion (Good) \\
            \hline
        \end{tabular}
    \end{table}

    \textbf{Cohesion Formula:}
    \begin{verbatim}
cohesionIndex = usedFields / totalFields
where usedFields = unique fields accessed by methods
    \end{verbatim}

    \subsubsection{Coupling Metrics}

    \begin{table}[H]
        \caption{Coupling Thresholds}
        \begin{tabular}{|l|l|l|}
            \hline
            \textbf{Dependencies} & \textbf{Level} & \textbf{Action} \\
            \hline
            < 5 & Low coupling & Acceptable \\
            \hline
            5 - 7 & Moderate coupling & Review \\
            \hline
            > 7 & High coupling & Refactor \\
            \hline
        \end{tabular}
    \end{table}

    \subsection{Complex Conditional Detector}

    \subsubsection{Detection Rules}

    \begin{table}[H]
        \caption{Complex Conditional Thresholds}
        \begin{tabular}{|l|l|}
            \hline
            \textbf{Condition} & \textbf{Threshold} \\
            \hline
            Logical operators in single condition & > 3 \\
            \hline
            Nested if statements & > 3 levels \\
            \hline
            Ternary operator complexity & > 2 nested \\
            \hline
        \end{tabular}
    \end{table}

    \subsection{Deficient Encapsulation Detector}

    \subsubsection{Detection Rules}

    \begin{itemize}[leftmargin=0.5in]
        \item Public non-final fields
        \item Protected fields (except in inheritance hierarchies)
        \item Package-private fields without justification
        \item Exceptions: public static final constants
    \end{itemize}

    \subsection{Long Parameter List Detector}

    \subsubsection{Thresholds}

    \begin{table}[H]
        \caption{Parameter List Thresholds}
        \begin{tabular}{|l|l|l|}
            \hline
            \textbf{Parameter Count} & \textbf{Severity} & \textbf{Recommendation} \\
            \hline
            <= 3 & None & Acceptable \\
            \hline
            4 - 6 & Low & Consider parameter object \\
            \hline
            7 - 9 & Medium & Refactor recommended \\
            \hline
            >= 10 & High & Refactor required \\
            \hline
        \end{tabular}
    \end{table}

    \subsection{Long Statement Detector}

    \subsubsection{Detection Criteria}

    \begin{table}[H]
        \caption{Statement Length Thresholds}
        \begin{tabular}{|l|l|}
            \hline
            \textbf{Characters} & \textbf{Action} \\
            \hline
            <= 120 & Acceptable \\
            \hline
            121 - 150 & Warning \\
            \hline
            > 150 & Critical \\
            \hline
        \end{tabular}
    \end{table}

    \subsection{Long Identifier Detector}

    \subsubsection{Naming Thresholds}

    \begin{table}[H]
        \caption{Identifier Length Thresholds}
        \begin{tabular}{|l|l|l|}
            \hline
            \textbf{Identifier Type} & \textbf{Max Length} & \textbf{Recommendation} \\
            \hline
            Variable & 30 characters & Use abbreviations \\
            \hline
            Method & 40 characters & Simplify name \\
            \hline
            Class & 35 characters & Reconsider design \\
            \hline
        \end{tabular}
    \end{table}

    \subsection{Missing Default Detector}

    \subsubsection{Detection Rules}

    \begin{itemize}[leftmargin=0.5in]
        \item Switch statements without default case
        \item Enum switches missing cases
        \item Exception: Complete enum coverage
    \end{itemize}

    \subsection{Unnecessary Abstraction Detector}

    \subsubsection{Detection Criteria}

    \begin{table}[H]
        \caption{Abstraction Detection Rules}
        \begin{tabular}{|l|l|}
            \hline
            \textbf{Pattern} & \textbf{Issue} \\
            \hline
            Interface with 1 implementation & Unnecessary abstraction \\
            \hline
            Abstract class with no subclasses & Unused abstraction \\
            \hline
            Single-method interface & Consider functional interface \\
            \hline
        \end{tabular}
    \end{table}

    \subsection{Memory Leak Detector}

    \subsubsection{Detection Patterns}

    \begin{table}[H]
        \caption{Memory Leak Patterns}
        \begin{tabular}{|l|l|}
            \hline
            \textbf{Pattern} & \textbf{Description} \\
            \hline
            Unclosed resources & Streams, connections not closed \\
            \hline
            Static collections & Growing static lists/maps \\
            \hline
            Event listeners & Registered but not removed \\
            \hline
            Thread locals & Not cleaned up \\
            \hline
            Inner class references & Implicit outer class reference \\
            \hline
        \end{tabular}
    \end{table}

    \subsection{Severity Classification}

    \begin{table}[H]
        \caption{Issue Severity Levels}
        \begin{tabular}{|l|p{8cm}|}
            \hline
            \textbf{Severity} & \textbf{Description} \\
            \hline
            Critical & Severe issues requiring immediate attention (security, memory leaks, major design flaws) \\
            \hline
            High & Important issues affecting maintainability (complex methods, high coupling) \\
            \hline
            Medium & Moderate issues affecting code quality (long parameters, missing defaults) \\
            \hline
            Low & Minor issues and suggestions (naming conventions, minor improvements) \\
            \hline
        \end{tabular}
    \end{table}

    \subsection{Grading Scale}

    \begin{table}[H]
        \caption{Code Quality Grading Scale}
        \begin{tabular}{|l|l|l|l|}
            \hline
            \textbf{Grade} & \textbf{Score} & \textbf{Issue Density} & \textbf{Quality Level} \\
            \hline
            A+ & 95-100 & < 0.5 issues/KLOC & Excellent \\
            \hline
            A & 90-94 & 0.5-1.0 issues/KLOC & Very Good \\
            \hline
            B & 80-89 & 1.0-2.0 issues/KLOC & Good \\
            \hline
            C & 70-79 & 2.0-5.0 issues/KLOC & Acceptable \\
            \hline
            D & 60-69 & 5.0-10.0 issues/KLOC & Below Average \\
            \hline
            F & 0-59 & > 10.0 issues/KLOC & Poor \\
            \hline
        \end{tabular}
    \end{table}



    \section{Installation and Deployment Guide}

    This appendix provides step-by-step instructions for installing, configuring, and deploying the DevSync platform.

    \subsection{Prerequisites}

    \subsubsection{Required Software}

    \begin{table}[H]
        \caption{Required Software Components}
        \begin{tabular}{|l|l|p{5cm}|}
            \hline
            \textbf{Software} & \textbf{Version} & \textbf{Purpose} \\
            \hline
            JDK & 21 or higher & Java runtime environment \\
            \hline
            Node.js & 18.x or higher & Frontend build tools \\
            \hline
            MySQL & 8.0 or higher & Database server \\
            \hline
            Maven & 3.8 or higher & Backend build tool \\
            \hline
            Git & Latest & Version control \\
            \hline
        \end{tabular}
    \end{table}

    \subsubsection{Optional Software}

    \begin{table}[H]
        \caption{Optional Software Components}
        \begin{tabular}{|l|l|p{5cm}|}
            \hline
            \textbf{Software} & \textbf{Version} & \textbf{Purpose} \\
            \hline
            Ollama & Latest & AI-powered analysis \\
            \hline
            Docker & Latest & Containerized deployment \\
            \hline
        \end{tabular}
    \end{table}

    \subsection{Backend Installation}

    \subsubsection{Step 1: Clone Repository}

    \begin{verbatim}
git clone https://github.com/your-repo/devsync.git
cd devsync
    \end{verbatim}

    \subsubsection{Step 2: Configure Database}

    Create MySQL database:
    \begin{verbatim}
mysql -u root -p
CREATE DATABASE devsyncdb;
CREATE USER 'devsync'@'localhost' IDENTIFIED BY 'password';
GRANT ALL PRIVILEGES ON devsyncdb.* TO 'devsync'@'localhost';
FLUSH PRIVILEGES;
EXIT;
    \end{verbatim}

    \subsubsection{Step 3: Configure Application Properties}

    Edit \texttt{src/main/resources/application.properties}:

    \begin{verbatim}
# Database Configuration
spring.datasource.url=jdbc:mysql://localhost:3306/devsyncdb
spring.datasource.username=devsync
spring.datasource.password=password
spring.jpa.hibernate.ddl-auto=update

# Server Configuration
server.port=8080

# File Upload Configuration
spring.servlet.multipart.max-file-size=50MB
spring.servlet.multipart.max-request-size=50MB

# Security Configuration
spring.security.user.name=admin
spring.security.user.password=admin123
    \end{verbatim}

    \subsubsection{Step 4: Build Backend}

    \begin{verbatim}
mvn clean install
    \end{verbatim}

    \subsubsection{Step 5: Run Backend}

    \begin{verbatim}
mvn spring-boot:run
    \end{verbatim}

    Or run the JAR file:
    \begin{verbatim}
java -jar target/devsync-0.0.1-SNAPSHOT.jar
    \end{verbatim}

    \subsection{Frontend Installation}

    \subsubsection{Step 1: Navigate to Frontend Directory}

    \begin{verbatim}
cd frontend
    \end{verbatim}

    \subsubsection{Step 2: Install Dependencies}

    \begin{verbatim}
npm install
    \end{verbatim}

    \subsubsection{Step 3: Configure Environment}

    Create \texttt{.env} file:

    \begin{verbatim}
VITE_API_URL=http://localhost:8080
VITE_APP_NAME=DevSync
    \end{verbatim}

    \subsubsection{Step 4: Run Development Server}

    \begin{verbatim}
npm run dev
    \end{verbatim}

    Access at: \texttt{http://localhost:5173}

    \subsubsection{Step 5: Build for Production}

    \begin{verbatim}
npm run build
    \end{verbatim}

    Production files will be in \texttt{dist/} directory.

    \subsection{Ollama AI Setup (Optional)}

    \subsubsection{Step 1: Install Ollama}

    Download from: \texttt{https://ollama.ai}

    \subsubsection{Step 2: Pull AI Model}

    \begin{verbatim}
ollama pull deepseek-coder:latest
    \end{verbatim}

    \subsubsection{Step 3: Start Ollama Service}

    \begin{verbatim}
ollama serve
    \end{verbatim}

    Service runs on: \texttt{http://localhost:11434}

    \subsection{Production Deployment}

    \subsubsection{Using Docker}

    Create \texttt{Dockerfile} for backend:

    \begin{verbatim}
FROM openjdk:21-jdk-slim
WORKDIR /app
COPY target/devsync-0.0.1-SNAPSHOT.jar app.jar
EXPOSE 8080
ENTRYPOINT ["java", "-jar", "app.jar"]
    \end{verbatim}

    Create \texttt{docker-compose.yml}:

    \begin{verbatim}
version: '3.8'
services:
  mysql:
    image: mysql:8.0
    environment:
      MYSQL_DATABASE: devsyncdb
      MYSQL_ROOT_PASSWORD: rootpass
    ports:
      - "3306:3306"

  backend:
    build: .
    ports:
      - "8080:8080"
    depends_on:
      - mysql
    environment:
      SPRING_DATASOURCE_URL: jdbc:mysql://mysql:3306/devsyncdb

  frontend:
    image: nginx:alpine
    volumes:
      - ./frontend/dist:/usr/share/nginx/html
    ports:
      - "80:80"
    \end{verbatim}

    Deploy:
    \begin{verbatim}
docker-compose up -d
    \end{verbatim}

    \subsubsection{Manual Deployment}

    \textbf{Backend:}
    \begin{enumerate}[leftmargin=0.5in]
        \item Build JAR: \texttt{mvn clean package}
        \item Copy JAR to server
        \item Run: \texttt{java -jar devsync.jar}
        \item Configure as system service (systemd/init.d)
    \end{enumerate}

    \textbf{Frontend:}
    \begin{enumerate}[leftmargin=0.5in]
        \item Build: \texttt{npm run build}
        \item Copy \texttt{dist/} to web server
        \item Configure Nginx/Apache
        \item Set up SSL certificate
    \end{enumerate}

    \subsection{Configuration}

    \subsubsection{Admin Settings}

    Initialize default settings via API:

    \begin{verbatim}
POST http://localhost:8080/api/admin/settings/init
    \end{verbatim}

    \subsubsection{File Upload Limits}

    Adjust in \texttt{application.properties}:

    \begin{verbatim}
spring.servlet.multipart.max-file-size=100MB
spring.servlet.multipart.max-request-size=100MB
    \end{verbatim}

    \subsubsection{CORS Configuration}

    Update in controllers:

    \begin{verbatim}
@CrossOrigin(origins = "https://your-domain.com")
    \end{verbatim}

    \subsection{Troubleshooting}

    \subsubsection{Common Issues}

    \begin{table}[H]
        \caption{Common Issues and Solutions}
        \begin{tabular}{|p{4cm}|p{7cm}|}
            \hline
            \textbf{Issue} & \textbf{Solution} \\
            \hline
            Database connection failed & Verify MySQL is running, check credentials \\
            \hline
            Port 8080 already in use & Change port in application.properties \\
            \hline
            File upload fails & Check file size limits and disk space \\
            \hline
            AI analysis not working & Verify Ollama is running on port 11434 \\
            \hline
            Frontend can't connect & Check CORS settings and API URL \\
            \hline
        \end{tabular}
    \end{table}

    \subsection{Maintenance}

    \subsubsection{Database Backup}

    \begin{verbatim}
mysqldump -u devsync -p devsyncdb > backup.sql
    \end{verbatim}

    \subsubsection{Log Files}

    Backend logs: \texttt{logs/spring.log}

    Frontend logs: Browser console

    \subsubsection{Updates}

    \begin{enumerate}[leftmargin=0.5in]
        \item Pull latest code: \texttt{git pull}
        \item Rebuild backend: \texttt{mvn clean install}
        \item Rebuild frontend: \texttt{npm run build}
        \item Restart services
    \end{enumerate}

    \subsection{Security Recommendations}

    \begin{itemize}[leftmargin=0.5in]
        \item Change default admin credentials
        \item Use strong database passwords
        \item Enable HTTPS in production
        \item Configure firewall rules
        \item Regular security updates
        \item Implement rate limiting
        \item Enable audit logging
    \end{itemize}

\end{document}
